\documentclass{scrartcl}
\usepackage[margin=3cm]{geometry}
\usepackage{amsmath}
\usepackage{amssymb}
\usepackage{amsthm}
\usepackage{blindtext}
\usepackage{datetime}
\usepackage{fontspec}
\usepackage{graphicx}
\usepackage{kotex}
\usepackage{mathrsfs}
\usepackage{mathtools}
\usepackage{pgf,tikz,pgfplots}

\pgfplotsset{compat=1.15}
\usetikzlibrary{arrows}

\newcommand\Overline[2][0.8pt]{%
  \begin{tikzpicture}[baseline=(a.base)]
    \node[inner xsep=0pt,inner ysep=1.5pt] (a) {$#2$};
    \draw[line width= #1] (a.north west) -- (a.north east);
  \end{tikzpicture}
}
\newtheorem{theorem}{Theorem}

\setmainhangulfont{Noto Serif CJK KR}[
  UprightFont=* Light, BoldFont=* Bold,
  Script=Hangul, Language=Korean, AutoFakeSlant,
]
\setsanshangulfont{Noto Sans CJK KR}[
  UprightFont=* DemiLight, BoldFont=* Medium,
  Script=Hangul, Language=Korean
]
\setmathhangulfont{Noto Sans CJK KR}[
  SizeFeatures={
    {Size=-6,  Font=* Medium},
    {Size=6-9, Font=*},
    {Size=9-,  Font=* DemiLight},
  },
  Script=Hangul, Language=Korean
]
\title{MATH311: Homework 10 (due May. 16)}
\author{손량(20220323)}
\date{Last compiled on: \today, \currenttime}

\newcommand{\un}[1]{\ensuremath{\ \mathrm{#1}}}
\newcommand{\imag}{\operatorname{Im}}
\newcommand{\real}{\operatorname{Re}}
\newcommand{\Log}{\operatorname{Log}}
\newcommand{\Arg}{\operatorname{Arg}}
\DeclareMathOperator*{\Res}{Res}

\begin{document}
\maketitle

\section{Section 5 \#19}
\subsection{Proof for (a)}
Fix \(\epsilon > 0\).
Since \(f'(0)\) exists, there exists some positive integer \(N\) such that \(n \geq N\) implies
\begin{align*}
  \left| \frac{f(\beta_n) - f(0)}{\beta_n} - f'(0) \right| < \frac{\epsilon}{2},\,
  \left| \frac{f(0) - f(\alpha_n)}{-\alpha_n} - f'(0) \right| < \frac{\epsilon}{2}
\end{align*}
Since \(\alpha_n < 0 < \beta_n\), we can write
\begin{align*}
  \frac{\beta_n}{\beta_n - \alpha_n} \left| \frac{f(\beta_n) - f(0)}{\beta_n} - f'(0) \right| < \frac{\epsilon}{2},\,
  \frac{-\alpha_n}{\beta_n - \alpha_n} \left| \frac{f(0) - f(\alpha_n)}{-\alpha_n} - f'(0) \right| < \frac{\epsilon}{2}
\end{align*}
By triangular inequality,
\begin{align*}
  \left| D_n - f'(0) \right|
  &= \left| \frac{\beta_n}{\beta_n - \alpha_n} \frac{f(\beta_n) - f(0)}{\beta_n} + \frac{-\alpha_n}{\beta_n - \alpha_n} \frac{f(0) - f(\alpha_n)}{-\alpha_n} - f'(0) \right| \\
  &\leq \frac{\beta_n}{\beta_n - \alpha_n} \left| \frac{f(\beta_n) - f(0)}{\beta_n} - f'(0) \right| + \frac{-\alpha_n}{\beta_n - \alpha_n} \left| \frac{f(0) - f(\alpha_n)}{-\alpha_n} - f'(0) \right| \\
  &< \frac{\epsilon}{2} + \frac{\epsilon}{2} < \epsilon
\end{align*}
so \(\lim_{n \to \infty} D_n = f'(0)\).

\subsection{Proof for (b)}
We can write
\begin{align}
  \nonumber D_n
  &= \frac{f(\beta_n) - f(0)}{\beta_n - \alpha_n} - \frac{f(\alpha_n) - f(0)}{\beta_n - \alpha_n} \\
  \nonumber &= \frac{f(\beta_n) - f(0)}{\beta_n} \cdot \frac{\beta_n}{\beta_n - \alpha_n} - \frac{f(\alpha_n) - f(0)}{\alpha_n} \cdot \frac{\alpha_n}{\beta_n - \alpha_n} \\
  \nonumber &= \frac{f(\beta_n) - f(0)}{\beta_n} \cdot \frac{\beta_n}{\beta_n - \alpha_n} - \frac{f(\alpha_n) - f(0)}{\alpha_n} \left( \frac{\beta}{\beta_n - \alpha_n} - 1 \right) \\
  \label{sec1_2_equality} &= \frac{f(\alpha_n) - f(0)}{\alpha_n} + \frac{\beta_n}{\beta_n - \alpha_n} \left( \frac{f(\beta_n) - f(0)}{\beta_n} - \frac{f(\alpha_n) - f(0)}{\alpha_n} \right)
\end{align}
Since \(\beta_n / (\beta_n - \alpha_n)\) is bounded, there exists some positive real number \(M\) such that \(\beta_n / (\beta_n - \alpha_n) \leq M\) and \(\beta_n > \alpha_n > 0\) so \(\beta_n / (\beta_n - \alpha_n) > 0\).
Using (\ref{sec1_2_equality}), we can write
\begin{align}\label{sec1_2_ineq}
  \frac{f(\alpha_n) - f(0)}{\alpha_n}
  < D_n
  \leq \frac{f(\alpha_n) - f(0)}{\alpha_n} + M\left( \frac{f(\beta_n) - f(0)}{\beta_n} - \frac{f(\alpha_n) - f(0)}{\alpha_n} \right)
\end{align}
Since \(f'(0)\) exists and \(a_n \to 0\) and \(b_n \to 0\) as \(n \to \infty\), we can take the limit of all sides of (\ref{sec1_2_ineq}) and obtain
\begin{align*}
  f'(0) \leq \lim_{n \to \infty} D_n \leq f'(0) + M (f'(0) - f'(0)) = f'(0)
\end{align*}
Thus, \(\lim_{n \to \infty} D_n = f'(0)\).

\subsection{Proof for (c)}
Fix \(\epsilon > 0\).
From continuity of \(f'(x)\), there exists some \(\delta > 0\) such that \(|x| < \delta\) implies \(|f'(x) - f'(0)| < \epsilon\).
Since \(\alpha_n \to 0\) and \(\beta_n \to 0\) as \(n \to \infty\), there exists some positive integer \(N\) such that \(n \geq N\) implies \(|\alpha_n| < \delta\) and \(|\beta_n| < \delta\).
By mean value theorem, for all \(n\) there exists \(\gamma_n\) between \(\alpha_n\) and \(\beta_n\) such that \(f'(\gamma_n) = D_n\).
Then, \(n \geq N\) implies \(|\gamma_n| < \delta\), hence \(|f'(\gamma_n) - f'(0)| < \epsilon\).
Thus, we can conclude that \(\lim_{n \to \infty} D_n = f'(0)\).

\section{Section 5 \#22}
\subsection{Proof for (a)}
Suppose that \(f(x)\) has two fixed points, \(\alpha\) and \(\beta\).
By mean value theorem, there exists \(\gamma\) between \(\alpha\) and \(\beta\) such that
\begin{align*}
  f'(\gamma) = \frac{f(\alpha) - f(\beta)}{\alpha - \beta} = \frac{\alpha - \beta}{\alpha - \beta} = 1
\end{align*}
which is a contradiction.

\subsection{Proof for (b)}
We can write
\begin{align*}
  f'(t) = 1 - \frac{e^t}{(1 + e^t)^2}
\end{align*}
Since \(e^t > 0\) for all \(t\), \(e^t < 1 + e^t < (1 + e^t)^2\), so \(0 < e^t / (1 + e^t)^2 < 1\) and \(0 < f'(t) < 1\).
Also, \((1 + e^t)^{-1} > 0\) for all \(t\) and \(f(t) > t\) and \(f\) does not have any fixed point.

\subsection{Proof for (c)}
Let \(g(t) = f(t) - t\).
Suppose that \(f(t)\) does not have any fixed point.
Then, Since \(f\) is differentiable, \(f\) is continuous and \(g\) is also continuous, and intermediate value theorem tells us that either \(g(t) > 0\) or \(g(t) < 0\) holds as \(g(t) \not = 0\) for all \(t\).
For all real \(t\), \(|f'(t)| = |g'(t) + 1| \leq A\) so \(-A - 1 \leq g'(t) \leq A - 1 < 0\).

Let's consider the case where \(g(t) > 0\) for all \(t\).
By mean value theorem, for all \(t > s\), we can write
\begin{align}
  \label{sec2_3_mvt} \frac{g(t) - g(s)}{t - s} &\leq A - 1 \\
  \nonumber g(t) &\leq (t - s)(A - 1) + g(s)
\end{align}
We can take \(t = s - g(s) / (A - 1)\), then
\begin{align*}
  g(t) \leq \left( s - \frac{g(s)}{A - 1} - s \right) (A - 1) + g(s) = 0
\end{align*}
which is a contradiction.

Now, let's consider the case where \(g(t) < 0\) for all \(t\).
Using the mean value theorem again, (\ref{sec2_3_mvt}) holds for all \(t > s\).
Then we can write
\begin{align*}
  g(s) \geq g(t) - (t - s)(A - 1)
\end{align*}
Taking \(s = t - g(t) / (A - 1)\),
\begin{align*}
  g(s) \geq g(t) - \left( t - t + \frac{g(t)}{A - 1} \right) (A - 1) = 0
\end{align*}
which is a contradiction.
In conclusion, \(f(t)\) has at least one fixed point.

Since \(x_{n + 1} = f(x_n)\), \(x_{n + 1} - x = f(x_n) - x = f(x_n) - f(x)\).
By mean value theorem, there exists \(c_n\) between \(x\) and \(x_n\) such that
\begin{align*}
  x_{n + 1} - x = f(x_n) - f(x) = (x_n - x) f'(c_n)
\end{align*}
so
\begin{align*}
  |x_{n + 1} - x| \leq A |x_n - x|
\end{align*}
by induction, we obtain
\begin{align*}
  |x_{n + 1} - x| \leq A^n |x_1 - x|
\end{align*}
and since \(0 \geq A < 1\), taking the limit of both sides gives us
\begin{align*}
  \lim_{n \to \infty} |x_{n + 1} - x| \leq \lim_{n \to \infty} A^n |x_1 - x| = 0
\end{align*}
so \(\lim_{n \to \infty} x_n = x\) by sandwich theorem and we get the desired result.

\subsection{Solution for (d)}
As \(x_{n + 1} = f(x_n)\), \((x_n, x_{n + 1}) = (x_n, f(x_n))\) is a point on \(y = f(x)\).
Thus, the process in (c) can be visualized by the path in the problem statement.

\section{Section 5 \#25}
\subsection{Solution for (a)}
A line tangent to the graph of \(f(x)\) at the point \((x_n, f(x_n))\) is \(y = f'(x_n) (x - x_n) + f(x_n)\).
From this, we know that the tangent line passes \((x_{n + 1}, 0)\).
Thus, we can interpret \(x_{n + 1}\) as the \(x\)-intercept of the tangent line of \(f(x)\) at the point \((x_n, f(x_n))\).

\subsection{Proof for (b)}
Suppose that \(x_k \in (\xi, b)\).
Mean value theorem tells us that there exists some \(c \in (\xi, x_n)\) such that
\begin{align}\label{sec3_2_mvt}
  f(\xi) - f(x_k) = f'(c) (\xi - x_k)
\end{align}
Since \(0 \leq f''(x) \leq M\) for all \(x \in [a, b]\), \(f'\) is a monotonically increasing function on \([a, b]\), so (\ref{sec3_2_mvt}) can be written as
\begin{align*}
  -f(x_k) \geq f'(x_k) (\xi - x_k)
\end{align*}
as \(f'(x) \leq f'(x_k)\).
Let \(l(x) = f'(x_k) (x - x_k) + f(x_k)\).
Then we can write
\begin{align*}
  l(\xi) = f'(x_k) (\xi - x_k) + f(x_k) \leq -f(x_k) + f(x_k) = 0
\end{align*}
If \(l(\xi) = 0\), \(x_{k + 1} = \xi\) and \(x_{k + 1} < x_k\).
If \(l(\xi) < 0\), since \(l(x)\) is continuous on \([\xi, x_k]\) intermediate value theorem tells us that there exists \(\alpha \in (\xi, x_k)\) such that \(l(\alpha) = 0\).
Then, \(x_{k + 1} = \alpha\) and \(x_{k + 1} < x_k\) holds.
Furthermore, \(\xi = \xi - f(\xi) / f'(\xi)\), so by induction \(x_n \in [\xi, b)\) implies \(x_{n + 1} \in [\xi, x_n)\) for all \(n = 1, 2, \dots\) and \(x_{n + 1} < x_n\).

Since \(\{x_n\}\) is monotonically decreasing and bounded, it converges to some value \(L\) by monotone convergence theorem.
If we take the limit of both sides of \(x_{n + 1} = x_n - f(x_n) / f'(x_n)\), we obtain \(L = L - f(L) / f'(L)\) and since \(L \in [\xi, x_1]\), \(f'(L) > 0\) so \(f(L) = 0\).
By definition, \(\xi\) is the only point in \((a, b)\) at which \(f(\xi) = 0\), so \(L = \xi\) and we get the desired result.

\subsection{Proof for (c)}
By Taylor's theorem, there exists \(t_n\) between \(\xi\) and \(x_n\) such that
\begin{align*}
  f(\xi) = f(x_n) + f'(x_n) (\xi - x_n) + \frac{f''(t_n)}{2} (\xi - x_n)^2
\end{align*}
then we can write
\begin{align*}
  -\frac{f(x_n)}{f'(x_n)} &= \xi - x_n + \frac{f''(t_n)}{2f'(x_n)} (\xi - x_n)^2 \\
  x_n -\frac{f(x_n)}{f'(x_n)} - \xi &= \frac{f''(t_n)}{2f'(x_n)} (\xi - x_n)^2 \\
  x_{n + 1} - \xi &= \frac{f''(t_n)}{2f'(x_n)} (x_n - \xi)^2
\end{align*}
and we get the desired result.

\subsection{Proof for (d)}
As we have proved in (b), \(x_n \in [\xi, b)\) for all \(n = 1, 2, \dots\) so \(x_{n + 1} - \xi \geq 0\).
Since \(f'(x) \geq \delta\) and \(f''(x) \leq M\) for all \(x \in [a, b]\), \(f''(t_n) / (2f'(x_n)) \leq M / (2\delta)\) for all \(n = 1, 2, \dots\).
Thus, we can write
\begin{align*}
  x_{n + 1} - \xi \leq \frac{M}{2\delta} (x_n - \xi)^2 = A(x_n - \xi)^2
\end{align*}
Let \(y_n = x_n - \xi\), then \(y_{n + 1} \leq Ay_n^2 \leq A(Ay_{n - 1}^2)^2 \leq \dots \leq A \cdot A^2 \cdot \dots \cdot A^{2^{n - 1}} y_1^{2^n} = A^{2^n - 1} y_1^{2^n} = [A(x_1 - \xi)]^{2^n} / A\) and we get the desired result.

\subsection{Proof for (e)}
Since \(f'(x) > 0\) for all \(x \in [a, b]\), \(g(x) - x = 0\) if and only if \(f(x) = 0\) and \(\xi\) is the only point in \((a, b)\) at which \(f(\xi) = 0\), so \(\xi\) is the only fixed point of \(g(x)\) in \([a, b]\).
Thus, Newton's method amounts to finding a fixed point of \(g\) in \([a, b]\).

We can write
\begin{align*}
  g'(x) = 1 - \frac{(f'(x))^2 - f(x) f''(x)}{(f'(x))^2} = 1 - \left( 1 - \frac{f(x) f''(x)}{(f'(x))^2} \right) = \frac{f(x) f''(x)}{(f'(x))^2}
\end{align*}
so \(g'(x)\) tends to 0 as \(x\) approaches \(\xi\).

\subsection{Solution for (f)}
We can write
\begin{align*}
  x_{n + 1} = x_n - \frac{f(x_n)}{f'(x_n)} = x_n - \frac{x_n^{1/3}}{x_n^{-2/3}/3} = -2x_n
\end{align*}
so \(x_n = (-2)^{n - 1} x_1\) and \(\{x_n\}\) does not converge as it is not bounded.

\section{Section 6 \#1}
Since \(f\) is bounded on \([a, b]\) and discontinuous only on \(x_0\) and \(\alpha\) is continuous there, we can use theorem~6.10 and know that \(f \in \mathscr{R}(\alpha)\).

Since \(f\) is zero except only one point, \(x_0\), for all partition \(P = \{p_0, \dots, p_n\}\) of \([a, b]\), \(m_i = \inf_{p_{i - 1} \leq x \leq p_i} f(x) = 0\) so \(L(P, f, \alpha) = \sum^n_{i = 1} m_i \Delta\alpha_i = 0\).
Thus, \(\sup L(P, f, \alpha) = \underline{\int^b_a} f d\alpha = 0\).
Since \(f \in \mathscr{R}(\alpha)\), \(\int^b_a f d\alpha = \underline{\int^b_a} f d\alpha = 0\).

\section{Section 6 \#2}
Suppose that \(f(\alpha) > 0\).
Since \(\int^b_a f dx = 0\), \(\underline{\int^b_a} f dx = \sup L(P, f) = 0\).
By definition, \(f(x) \geq 0\) so for all partition \(P = \{x_0, \dots, x_n\}\) of \([a, b]\), \(L(P, f) \geq 0\) as \(m_i = \inf_{x_{i - 1} \leq x \leq x_i} f(x) \geq 0\), which implies \(L(P, f) = \sum^n_{i = 1} m_i \Delta x_i \geq 0\).
Then, since \(\sup L(P, f) = 0\), \(L(P, f) = 0\) for all partition \(P\) of \([a, b]\).
Since \(f\) is continuous at \(\alpha\), there exists some \(\delta > 0\) such that \((\alpha - \delta, \alpha + \delta) \in [a, b]\) and \(|x - \alpha| < \delta\) implies \(|f(x) - f(\alpha)| < f(\alpha) / 2\).
Then \(f(x) > f(\alpha) / 2\) for all \(x \in (\alpha - \delta, \alpha + \delta)\).
Consider a partition \(P = \{a, \alpha - \delta / 2, \alpha + \delta / 2, b\}\).
We can write
\begin{align*}
  L(P, f) \geq \delta \inf_{\alpha - \frac{\delta}{2} \leq x \leq \alpha + \frac{\delta}{2}} f(x) \geq \frac{\delta f(\alpha)}{2} > 0
\end{align*}
and it is a contradiction.

\section{Section 6 \#3}
\subsection{Proof for (a)}
Suppose that \(f \in \mathscr{R}(\beta_1)\).
Fix \(\epsilon > 0\).
There exists some partition \(P = \{p_0, \dots, p_n\}\) of \([-1, 1]\) such that \(U(P, f, \beta_1) - L(P, f, \beta_1) < \epsilon\).
Let \(P' = P \cup \{0\}\), then \(P'\) is a refinement of \(P\) so \(U(P, f, \beta_1) \geq U(P', f, \beta_1) \geq L(P', f, \beta_1) \geq L(P, f, \beta_1)\) and \(U(P', f, \beta_1) - L(P', f, \beta_1) < \epsilon\).
Suppose that \(p_k \leq 0 < p_{k + 1}\), then
\begin{align*}
  L(P', f, \beta_1)
  &= \sum^{k - 1}_{i = 0} (\beta_1(p_{i + 1}) - \beta_1(p_i)) \inf_{x \in [p_i, p_{i + 1}]} f(x) \\
  &+ \sum^{n - 1}_{i = k + 1} (\beta_1(p_{i + 1}) - \beta_1(p_i)) \inf_{x \in [p_i, p_{i + 1}]} f(x) \\
  &+ (\beta_1(0) - \beta_1(p_k)) \inf_{x \in [p_k, 0]} f(x)
  + (\beta_1(p_{k + 1}) - \beta_1(0)) \inf_{x \in [0, p_{k + 1}]} f(x) \\
  &= \inf_{x \in [0, p_{k + 1}]} f(x)
\end{align*}
and we can also write
\begin{align*}
  U(P', f, \beta_1)
  &= \sum^{k - 1}_{i = 0} (\beta_1(p_{i + 1}) - \beta_1(p_i)) \sup_{x \in [p_i, p_{i + 1}]} f(x) \\
  &+ \sum^{n - 1}_{i = k + 1} (\beta_1(p_{i + 1}) - \beta_1(p_i)) \sup_{x \in [p_i, p_{i + 1}]} f(x) \\
  &+ (\beta_1(0) - \beta_1(p_k)) \sup_{x \in [p_k, 0]} f(x)
  + (\beta_1(p_{k + 1}) - \beta_1(0)) \sup_{x \in [0, p_{k + 1}]} f(x) \\
  &= \sup_{x \in [0, p_{k + 1}]} f(x) \\
\end{align*}
so
\begin{align*}
  U(P', f, \beta_1) - L(P', f, \beta_1)
  = \sup_{x \in [0, p_{k + 1}]} f(x) - \inf_{x \in [0, p_{k + 1}]} f(x) < \epsilon
\end{align*}
From this, we know that \(0 < x < p_{k + 1}\) implies \(|f(x) - f(0)| < \epsilon\) so \(f(0+) = f(0)\).
Also, we know that
\begin{align*}
  L(P, f, \beta_1) \leq L(P', f, \beta_1) \leq f(0) \leq U(P', f, \beta_1) \leq U(P, f, \beta_1)
\end{align*}
for all partition \(P\) of \([-1, 1]\) and \(P' = P \cup \{0\}\), so
\begin{align*}
  \underline{\int^1_{-1}} f d\beta_1 \leq f(0) \leq \overline{\int^1_{-1}} f d\beta_1
\end{align*}
and since \(f \in \mathscr{R}(\beta_1)\), \(\underline{\int^1_{-1}} f d\beta_1 = \overline{\int^1_{-1}} f d\beta_1 = \int^1_{-1} f d\beta_1 = f(0)\).

Suppose that \(f(0+) = f(0)\).
Fix \(\epsilon > 0\).
There exists some \(\delta > 0\) such that \(0 < x < \delta\) implies \(|f(x) - f(0)| < \epsilon / 4\).
Consider \(P = \{-1, 0, \delta / 2, 1\}\), a partition of \([-1, 1]\).
Then we can write
\begin{align*}
  L(P, f, \beta_1) &= (\beta_1(0) - \beta_1(-1)) \inf_{x \in [-1, 0]} f(x) + (\beta_1(\delta / 2) - \beta_1(0)) \inf_{x \in [0, \delta / 2]} f(x) \\
                   &+ (\beta_1(1) - \beta_1(\delta / 2)) \inf_{x \in [\delta / 2, 1]} f(x) \\
                   &= (\beta_1(\delta / 2) - \beta_1(0)) \inf_{x \in [0, \delta / 2]} f(x) = \inf_{x \in [0, \delta / 2]} f(x) \\
  U(P, f, \beta_1) &= (\beta_1(0) - \beta_1(-1)) \sup_{x \in [-1, 0]} f(x) + (\beta_1(\delta / 2) - \beta_1(0)) \sup_{x \in [0, \delta / 2]} f(x) \\
                   &+ (\beta_1(1) - \beta_1(\delta / 2)) \sup_{x \in [\delta / 2, 1]} f(x) \\
                   &= (\beta_1(\delta / 2) - \beta_1(0)) \sup_{x \in [0, \delta / 2]} f(x) = \sup_{x \in [0, \delta / 2]} f(x)
\end{align*}
and
\begin{align*}
  U(P, f, \beta_1) - L(P, f, \beta_1)
  &= \sup_{x \in [0, \delta / 2]} f(x) - \inf_{x \in [0, \delta / 2]} f(x) \\
  &\leq \left( f(0) + \frac{\epsilon}{4} \right) - \left( f(0) - \frac{\epsilon}{4} \right)
  = \frac{\epsilon}{2}
  < \epsilon
\end{align*}
Since the choice of \(\epsilon\) is arbitrary, we can conclude that \(f \in \mathscr{R}(\beta_1)\).

\subsection{Proof for (b)}
Let's prove that \(f \in \mathscr{R}(\beta_2)\) if and only if \(f(0-) = f(0)\).

Suppose that \(f \in \mathscr{R}(\beta_2)\).
Fix \(\epsilon > 0\).
There exists some partition \(P = \{p_0, \dots, p_n\}\) of \([-1, 1]\) such that \(U(P, f, \beta_2) - L(P, f, \beta_2) < \epsilon\).
Let \(P' = P \cup \{0\}\), then \(P'\) is a refinement of \(P\) so \(U(P, f, \beta_2) \geq U(P', f, \beta_2) \geq L(P', f, \beta_2) \geq L(P, f, \beta_2)\) and \(U(P', f, \beta_2) - L(P', f, \beta_2) < \epsilon\).
Suppose that \(p_k < 0 \leq p_{k + 1}\), then
\begin{align*}
  L(P', f, \beta_2)
  &= \sum^{k - 1}_{i = 0} (\beta_2(p_{i + 1}) - \beta_2(p_i)) \inf_{x \in [p_i, p_{i + 1}]} f(x) \\
  &+ \sum^{n - 1}_{i = k + 1} (\beta_2(p_{i + 1}) - \beta_2(p_i)) \inf_{x \in [p_i, p_{i + 1}]} f(x) \\
  &+ (\beta_2(0) - \beta_2(p_k)) \inf_{x \in [p_k, 0]} f(x)
  + (\beta_2(p_{k + 1}) - \beta_2(0)) \inf_{x \in [0, p_{k + 1}]} f(x) \\
  &= \inf_{x \in [p_k, 0]} f(x)
\end{align*}
and we can also write
\begin{align*}
  U(P', f, \beta_2)
  &= \sum^{k - 1}_{i = 0} (\beta_2(p_{i + 1}) - \beta_2(p_i)) \sup_{x \in [p_i, p_{i + 1}]} f(x) \\
  &+ \sum^{n - 1}_{i = k + 1} (\beta_2(p_{i + 1}) - \beta_2(p_i)) \sup_{x \in [p_i, p_{i + 1}]} f(x) \\
  &+ (\beta_2(0) - \beta_2(p_k)) \sup_{x \in [p_k, 0]} f(x)
  + (\beta_2(p_{k + 1}) - \beta_2(0)) \sup_{x \in [0, p_{k + 1}]} f(x) \\
  &= \sup_{x \in [p_k, 0]} f(x)
\end{align*}
so
\begin{align*}
  U(P', f, \beta_2) - L(P', f, \beta_2)
  = \sup_{x \in [p_k, 0]} f(x) - \inf_{x \in [p_k, 0]} f(x) < \epsilon
\end{align*}
From this, we know that \(p_k < x < 0\) implies \(|f(x) - f(0)| < \epsilon\) so \(f(0-) = f(0)\).
Also, we know that
\begin{align*}
  L(P, f, \beta_2) \leq L(P', f, \beta_2) \leq f(0) \leq U(P', f, \beta_2) \leq U(P, f, \beta_2)
\end{align*}
for all partition \(P\) of \([-1, 1]\) and \(P' = P \cup \{0\}\), so
\begin{align*}
  \underline{\int^1_{-1}} f d\beta_2 \leq f(0) \leq \overline{\int^1_{-1}} f d\beta_2
\end{align*}
and since \(f \in \mathscr{R}(\beta_2)\), \(\underline{\int^1_{-1}} f d\beta_2 = \overline{\int^1_{-1}} f d\beta_2 = \int^1_{-1} f d\beta_2 = f(0)\).

Suppose that \(f(0-) = f(0)\).
Fix \(\epsilon > 0\).
There exists some \(\delta > 0\) such that \(-\delta < x < 0\) implies \(|f(x) - f(0)| < \epsilon / 4\).
Consider \(P = \{-1, -\delta / 2, 0, 1\}\), a partition of \([-1, 1]\).
Then we can write
\begin{align*}
  L(P, f, \beta_2) &= (\beta_2(-\delta / 2) - \beta_2(-1)) \inf_{x \in [-1, -\delta / 2]} f(x) + (\beta_2(0) - \beta_2(-\delta / 2)) \inf_{x \in [-\delta / 2, 0]} f(x) \\
                   &+ (\beta_2(1) - \beta_2(0)) \inf_{x \in [0, 1]} f(x) \\
                   &= (\beta_2(0) - \beta_2(-\delta / 2)) \inf_{x \in [-\delta / 2, 0]} f(x) = \inf_{x \in [-\delta / 2, 0]} f(x) \\
  U(P, f, \beta_2) &= (\beta_2(-\delta / 2) - \beta_2(-1)) \sup_{x \in [-1, -\delta / 2]} f(x) + (\beta_2(0) - \beta_2(-\delta / 2)) \sup_{x \in [-\delta / 2, 0]} f(x) \\
                   &+ (\beta_2(1) - \beta_2(0)) \sup_{x \in [0, 1]} f(x) \\
                   &= (\beta_2(0) - \beta_2(-\delta / 2)) \sup_{x \in [-\delta / 2, 0]} f(x) = \sup_{x \in [-\delta / 2, 0]} f(x) \\
\end{align*}
and
\begin{align*}
  U(P, f, \beta_2) - L(P, f, \beta_2)
  &= \sup_{x \in [-\delta / 2, 0]} f(x) - \inf_{x \in [-\delta / 2, 0]} f(x) \\
  &\leq \left( f(0) + \frac{\epsilon}{4} \right) - \left( f(0) - \frac{\epsilon}{4} \right)
  = \frac{\epsilon}{2}
  < \epsilon
\end{align*}
Since the choice of \(\epsilon\) is arbitrary, we can conclude that \(f \in \mathscr{R}(\beta_2)\).

\subsection{Proof for (c)}
Suppose that \(f \in \mathscr{R}(\beta_3)\).
Fix \(\epsilon > 0\).
There exists some partition \(P = \{p_0, \dots, p_{n - 1}\}\) of \([-1, 1]\) such that \(U(P, f, \beta_3) - L(P, f, \beta_3) < \epsilon / 2\).
Let \(P' = P \cup \{0\} = \{p'_0, \dots, p'_n\}\), then \(P'\) is a refinement of \(P\) so \(U(P, f, \beta_3) \geq U(P', f, \beta_3) \geq L(P', f, \beta_3) \geq L(P, f, \beta_3)\) and \(U(P', f, \beta_3) - L(P', f, \beta_3) < \epsilon / 2\).
Suppose that \(p'_k = 0\), then
\begin{align*}
  L(P', f, \beta_3)
  &= \sum^{n - 1}_{i = 0} (\beta_3(p'_{i + 1}) - \beta_3(p'_i)) \inf_{x \in [p'_i, p'_{i + 1}]} f(x) \\
  &= (\beta_3(0) - \beta_3(p'_{k - 1})) \inf_{x \in [p'_{k - 1}, 0]} f(x) + (\beta_3(p'_{k + 1}) - \beta_3(0)) \inf_{x \in [0, p'_{k + 1}]} f(x) \\
  &= \frac{1}{2} \inf_{x \in [p'_{k - 1}, 0]} f(x) + \frac{1}{2} \inf_{x \in [0, p'_{k + 1}]} f(x)
\end{align*}
and we can also write
\begin{align*}
  U(P', f, \beta_3)
  &= \sum^{n - 1}_{i = 0} (\beta_3(p'_{i + 1}) - \beta_3(p'_i)) \sup_{x \in [p'_i, p'_{i + 1}]} f(x) \\
  &= (\beta_3(0) - \beta_3(p'_{k - 1})) \sup_{x \in [p'_{k - 1}, 0]} f(x) + (\beta_3(p'_{k + 1}) - \beta_3(0)) \sup_{x \in [0, p'_{k + 1}]} f(x) \\
  &= \frac{1}{2} \sup_{x \in [p'_{k - 1}, 0]} f(x) + \frac{1}{2} \sup_{x \in [0, p'_{k + 1}]} f(x)
\end{align*}
so
\begin{align*}
  U(P', f, \beta_3) - L(P', f, \beta_3)
  &= \frac{1}{2} \left( \sup_{x \in [p'_{k - 1}, 0]} f(x) - \inf_{x \in [p'_{k - 1}, 0]} f(x) \right) \\
  &+ \frac{1}{2} \left( \sup_{x \in [0, p'_{k + 1}]} f(x) - \inf_{x \in [0, p'_{k + 1}]} f(x) \right) < \epsilon / 2
\end{align*}
then we can write
\begin{align*}
  \sup_{x \in [p'_{k - 1}, 0]} f(x) - \inf_{x \in [p'_{k - 1}, 0]} f(x) &< \epsilon \\
  \sup_{x \in [0, p'_{k + 1}]} f(x) - \inf_{x \in [0, p'_{k + 1}]} f(x) &< \epsilon
\end{align*}
We can take \(\delta = \min \{|p'_{k - 1}|, |p'_{k + 1}|\}\) then \(0 < |x| < \delta\) implies \(|f(x) - f(0)| < \epsilon\) so \(\lim_{x \to 0} f(x) = f(0)\) and \(f\) is continuous at \(x = 0\).

Suppose that \(f\) is continuous at \(x = 0\).
Fix \(\epsilon > 0\).
There exists some \(\delta > 0\) such that \(|x| < \delta\) implies \(|f(x) - f(0)| < \epsilon / 4\).
Consider \(P = \{-1, -\delta / 2, 0, \delta / 2, 1\}\), a partition of \([-1, 1]\).
Then we can write
\begin{align*}
  L(P, f, \beta_3) &= (\beta_3(-\delta / 2) - \beta_3(-1)) \inf_{x \in [-1, -\delta / 2]} f(x) + (\beta_3(0) - \beta_3(-\delta / 2)) \inf_{x \in [-\delta / 2, 0]} f(x) \\
                   &+ (\beta_3(\delta / 2) - \beta_3(0)) \inf_{x \in [0, \delta / 2]} f(x) + (\beta_3(1) - \beta_3(\delta / 2)) \inf_{x \in [\delta / 2, 1]} f(x) \\
                   &= \frac{1}{2} \inf_{x \in [-\delta / 2, 0]} f(x) + \frac{1}{2} \inf_{x \in [0, \delta / 2]} f(x) \\
  U(P, f, \beta_3) &= (\beta_3(-\delta / 2) - \beta_3(-1)) \sup_{x \in [-1, -\delta / 2]} f(x) + (\beta_3(0) - \beta_3(-\delta / 2)) \sup_{x \in [-\delta / 2, 0]} f(x) \\
                   &+ (\beta_3(\delta / 2) - \beta_3(0)) \sup_{x \in [0, \delta / 2]} f(x) + (\beta_3(1) - \beta_3(\delta / 2)) \sup_{x \in [\delta / 2, 1]} f(x) \\
                   &= \frac{1}{2} \sup_{x \in [-\delta / 2, 0]} f(x) + \frac{1}{2} \sup_{x \in [0, \delta / 2]} f(x) \\
\end{align*}
and
\begin{align*}
  U(P, f, \beta_3) - L(P, f, \beta_3)
  &= \frac{1}{2} \left( \sup_{x \in [-\delta / 2, 0]} f(x) - \inf_{x \in [-\delta / 2, 0]} f(x) \right) \\
  &+ \frac{1}{2} \left( \sup_{x \in [0, \delta / 2]} f(x) - \inf_{x \in [0, \delta / 2]} f(x) \right) \\
  &\leq 2 \cdot \frac{1}{2} \left[ \left( f(0) + \frac{\epsilon}{4} \right) - \left( f(0) - \frac{\epsilon}{4} \right) \right]
  = \frac{\epsilon}{2}
  < \epsilon
\end{align*}
Since the choice of \(\epsilon\) is arbitrary, we can conclude that \(f \in \mathscr{R}(\beta_3)\).

\subsection{Proof for (d)}
Since \(f\) is continuous at \(x = 0\), \(f(0-) = f(0) = f(0+)\) so by the result we have proven earlier,
\begin{align*}
  \int f d\beta_2 = f(0)
\end{align*}
Fix \(\epsilon > 0\).
\(f \in \mathscr{R}(\beta_3)\) implies that there exists some partition \(P = \{p_0, \dots, p_{n - 1}\}\) of \([-1, 1]\) such that \(U(P, f, \beta_3) - L(P, f, \beta_3) < \epsilon / 2\). Let \(P' = P \cup \{0\} = \{p'_0, \dots, p'_n\}\) then \(P'\) is a refinement of \(P\) so \(U(P, f, \beta_3) \geq U(P', f, \beta_3) \geq L(P', f, \beta_3) \geq L(P, f, \beta_3)\) and \(U(P', f, \beta_3) - L(P', f, \beta_3) < \epsilon / 2\).
From the proof for (c), we know that
\begin{align*}
  L(P, f, \beta_3) \leq L(P', f, \beta_3) \leq f(0) \leq U(P', f, \beta_3) \leq U(P, f, \beta_3)
\end{align*}
for all partition \(P\) of \([-1, 1]\) and \(P' = P \cup \{0\}\), so
\begin{align*}
  \underline{\int^1_{-1}} f d\beta_3 \leq f(0) \leq \overline{\int^1_{-1}} f d\beta_3
\end{align*}
and since \(f \in \mathscr{R}(\beta_3)\), \(\underline{\int^1_{-1}} f d\beta_3 = \overline{\int^1_{-1}} f d\beta_3 = \int^1_{-1} f d\beta_3 = f(0)\) and we get the desired result.

\end{document}
