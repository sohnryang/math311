\documentclass{scrartcl}
\usepackage[margin=3cm]{geometry}
\usepackage{amsmath}
\usepackage{amssymb}
\usepackage{amsthm}
\usepackage{blindtext}
\usepackage{datetime}
\usepackage{fontspec}
\usepackage{graphicx}
\usepackage{kotex}
\usepackage{mathrsfs}
\usepackage{mathtools}
\usepackage{pgf,tikz,pgfplots}

\pgfplotsset{compat=1.15}
\usetikzlibrary{arrows}
\newtheorem{theorem}{Theorem}

\setmainhangulfont{Noto Serif CJK KR}[
  UprightFont=* Light, BoldFont=* Bold,
  Script=Hangul, Language=Korean, AutoFakeSlant,
]
\setsanshangulfont{Noto Sans CJK KR}[
  UprightFont=* DemiLight, BoldFont=* Medium,
  Script=Hangul, Language=Korean
]
\setmathhangulfont{Noto Sans CJK KR}[
  SizeFeatures={
    {Size=-6,  Font=* Medium},
    {Size=6-9, Font=*},
    {Size=9-,  Font=* DemiLight},
  },
  Script=Hangul, Language=Korean
]
\title{Homework 2 (due Mar. 7)}
\author{손량(20220323)}
\date{Last compiled on: \today, \currenttime}

\newcommand{\un}[1]{\ensuremath{\ \mathrm{#1}}}

\begin{document}
\maketitle

\section{Problem 6}
\subsection{Solution for (a)}
If \(m = 0\) and \(p = 0\), \(b^m = b^p = 1\) and the equality holds.
Let's prove for \(m, p \not = 0\) case.

Let \(x := (b^m)^{1 / n}, y := (b^p)^{1 / q}\). We can obtain

\[
  x^{np} = (x^n)^p = (b^m)^p = b^{mp},\quad y^{mq} = (y^q)^m = (b^p)^m = b^{mp}
\]

\(x, y\) can be written as follows, by the theorem 1.21 in the book.

\[
  x = (b^{mp})^{1 / (np)}, \quad y = (b^{mp})^{1 / (mq)}
\]

Since \(np = mq\), \(x = (b^{mp})^{1 / (np)} = (b^{mp})^{1 / (mq)} = y\) holds.

\subsection{Solution for (b)}
Let \(r := m / n, s = p / q\). \(b^{r + s}\) can be written as:

\[
  b^{r + s} = b^{(mq + np) / (nq)} = (b^{mq + np})^{1 / (nq)}
\]

By corollary of theorem 1.21,

\[
  (b^{mq} b^{np})^{1 / (nq)} = (b^{mq})^{1 / (nq)} (b^{np})^{1 / (nq)}
\]

Using the fact we proved in (a),

\[
  (b^{mq})^{1 / (nq)} (b^{np})^{1 / (nq)} = (b^m)^{1 / n} (b^p)^{1 / q} = b^r b^s
\]

\subsection{Solution for (c)}
First, let's show that \(b^r\) is an upper bound of \(B(r)\).
Suppose that there exists an element \(b^u\) such that \(u \leq r\) and \(b^u > b^r\).
Using the result from (b),

\[
  b^r = b^{r - u + u} = b^{r - u} b^u < b^u
\]

If we write \(u = m / n\) for integers \(m\) and \(n\), it can be shown that \(b^u = (b^m)^{1 / n} > 0\) using the theorem 1.21 in the book and the fact that \(b^m > 0\).
Thus, \(b^{r - u} < 1\) should hold. Since \(r \geq u\), \(r - u\) can be written as \(p / q\) where \(p\) is a nonnegative integer, and \(q\) is a positive integer.
Then \(b^{r - u} = (b^p)^{1 / q} > 0\), so \(0 < (b^p)^{1 / q} < 1\) implies \(b^p < 1\).
However, since \(b > 1\), it implies \(b^p \geq 1\) for nonnegative integer \(p\), which is a contradiction.
Thus, \(b^u\) cannot exist.

Since \(B(r)\) is a set of reals with an upper bound, it has the least upper bound, \(\sup B(r)\).
Suppose that there exists \(b^v\) where \(v\) is rational, such that \(b^v < b^r\) and \(b^v \geq y \; \forall y \in B(r)\).
It is evident that such \(b^v\) cannot exist since \(b^r\) is also an element of \(B(r)\).
In conclusion, no upper bound smaller than \(b^r\) cannot exist and \(b^r = \sup B(r)\).

\subsection{Solution for (d)}
First, let's show that \(\sup B(x)\sup B(y)\) is an upper bound of \(B(x + y)\).
For all elements in \(b^t \in B(x + y)\) where \(t \leq x + y\) is a rational, there are two possibilities:

\begin{enumerate}
  \item \(t < x + y\)
  \item \(t = x + y\)
\end{enumerate}

If there exists an element \(b^t\) such that \(t = x + y\), \(x + y\) is a rational and \(\sup B(x + y) = b^{x + y}\) as we proved in (c).
If there are only elements \(b^t\) such that \(t < x + y\), there exists a rational \(r\) such that \(t - y < r < x\) by theorem 1.20.
Then, we can let \(s = t - r\) and \(s < y\) holds.
In other words, \(t\) can be written as \(r + s\) where \(r, s\) are rationals such that \(r < x\) and \(s < y\).
As proven in (b), \(b^t = b^{r + s} = b^r b^s\) and \(b^r \in B(x), b^s \in B(y)\) holds.
By the definition in (c), \(b^r \leq \sup B(x) = b^x\) and \(b^s \leq \sup B(y) = b^y\).
This implies that for all \(b^t \in B(x + y)\),

\[
  b^t = b^{r + s} = b^r b^s \leq \sup B(x) \sup B(y)
\]

Thus, \(\sup B(x) \sup B(y)\) is an upper bound of \(B(x + y)\).
Now we have to show that \(\sup B(x + y) = \sup B(x) \sup B(y)\).
Suppose that there exists an upper bound of \(B(x + y)\) that is smaller than \(\sup B(x) \sup B(y)\) and call it \(c\).
The following holds:

\[
  \frac{c}{\sup B(x)} < \sup B(y)
\]

Since \(\sup B(y)\) is the least upper bound of \(B(y)\), there exists \(b^v \in B(y)\) such that \(c / \sup B(x) < b^v \leq \sup B(y)\).
In the same vein, there exists \(b^u \in B(x)\) such that \(c / b^v < b^u \leq \sup B(x)\).
Now, we get \(c < b^u b^v\) and \(b^u b^v \in B(x + y)\) which is a contradiction.
Such lower bound \(c\) does not exist, so \(\sup B(x) \sup B(y) = \sup B(x + y)\) and \(b^{x + y} = b^x b^y\).

\section{Problem 7}
\subsection{Solution for (a)}
\(b^n - 1\) can be written as follows:

\[
  b^n - 1 = (b - 1)(b^{n - 1} + b^{n - 2} + \dots + b + 1)
\]

Since \(b > 1\), we know \(b^{n - 1} > b^{n - 2} > \dots > b > 1\).
The polynomial \(b^{n - 1} + b^{n - 2} + \dots + b + 1\) has \(n\) terms, and each term is greater or equal to 1.
So we get

\[
  b^i \geq 1 (i = 0, 1, \dots, n - 1) \Longrightarrow b^{n - 1} + b^{n - 2} + \dots + b + 1 \geq n
\]

And we obtain the inequality \(b^n - 1 \geq n(b - 1)\).

\subsection{Solution for (b)}
Let \(c := b^{1 / n}\). By theorem 1.21, \(c > 0\). Suppose that \(c \leq 1\).
Then \(1 \geq c \geq c^2 \geq \dots \geq c^n = (b^{1 / n})^n = b\), and it is a contradiction.
Hence, \(c > 1\). Plugging \(c\) to the inequality we obtained in (a), we get

\[
  c^n - 1 \geq n(c - 1) \Longleftrightarrow b - 1 \geq n(b^{1 / n} - 1)
\]

\subsection{Solution for (c)}
Since \(t > 1\), \(n > (b - 1) / (t - 1)\) can be written as \(n(t - 1) > b - 1\).
By the inequality from (b), we can write

\[
  n(t - 1) > b - 1 \geq n(b^{1 / n} - 1)
\]

Since \(n\) is positive, \(b^{1 / n} < t\) holds.

\subsection{Solution for (d)}
Since \(b > 0\), \(b^w > 0\) and \(t := y \cdot b^{-w} > 1\) holds.
By the archimedean property, there exists a positive interger \(n\) such that \(n(t - 1) > b - 1\).
From the result from (c), \(b^{1 / n} < t = y \cdot b^{-w}\) and \(b^{w + (1 / n)} < y\) holds for a sufficiently large integer \(n\).

\subsection{Solution for (e)}
Since \(y > 0\), \(t := b^w / y > 1\) holds.
By the archimedean property, there exists a positive integer \(n\) such that \(n(t - 1) > b - 1\).
From the result from (c), \(b^{1 / n} < t = b^w / y\) and \(b^{w - (1 / n)} < y\) holds for a sufficiently large integer \(n\).

\subsection{Solution for (f)}
Suppose that \(b^x > y\).
By the result from (e), there exists a positive integer \(n\) such that \(b^{x - (1 / n)} > y\).
This means that \(x - (1 / n)\) is also an upper bound of \(A\), and since \(b^{x - (1 / n)} < b^x\), it is a contradiction.
Thus, \(b^x \leq y\) holds.

Suppose that \(b^x < y\).
By the result from (d), there exists a positive integer \(n\) such that \(b^{x + (1 / n)} < y\).
This means that \(x\) cannot be an upper bound of \(A\), because there exists an \(b^{x + (1 / n)}\) is also an element of \(A\).
Thus, \(b^x \geq y\) holds.
In conclusion, \(b^x\) should satisfy both \(b^x \leq y\) and \(b^x \geq y\), so \(b^x = y\).

\subsection{Solution for (g)}
Suppose that there exists a real \(z \not = x\) such that \(b^z = y\).
There are two possibilities:

\begin{enumerate}
  \item \(z > x\)
  \item \(z < x\)
\end{enumerate}

In \(z > x\) case, \(b^z = b^{z - x + x} = b^{z - x} b^x > b^x = y\), so it is a contradiction.

In \(z < x\) case, \(b^x = b^{x - z + z} = b^{x - z} b^z > b^z = y\), so it is also a contradiction. In conclusion, such \(z\) cannot exist and thus \(x\) is unique.

\section{Problem 8}
Suppose that a relation \(<\) is defined for complex field, and it satsifies all axioms for ordered field. Then, one of the statements is true.

\[
  i < 0, \quad i = 0, \quad i > 0
\]

\(i = 0\) is impossible because \(i \cdot i = -1 \not = 0\), by definition.

If \(i < 0\), we can multiply both sides with \(i\) and obtain \(-1 > 0\).
Multiplying both sides with \(i\) agian, we obtain \(-i < 0\), which contradicts with \(i < 0\).

If \(i > 0\), the same operations can be done like the \(i < 0\) case, and we obtain \(-i > 0\), which also contradicts with \(i > 0\).
In conclusion, the assumed relation \(<\) cannot exist.

\section{Problem 20}
\subsection{Proof for the least-upper-bound property}
Let \(A\) be a nonempty subset of \(\mathbb{R}\) and assume that \(\beta \in \mathbb{R}\) is an upper bound of \(A\).
Define \(\gamma\) as the union of all \(\alpha \in A\).
In other words, \(p \in \gamma\) if and only if \(p \in \alpha\) for some \(\alpha \in A\).
Let's prove that \(\gamma \in \mathbb{R}\) and \(y = \sup A\).

Since \(A\) is nonempty, there exists a nonempty \(\alpha_0 \in A\).
From \(\alpha_0 \subset \gamma\), \(\gamma\) is not empty.
For all \(\alpha \in A\), \(\alpha \subset \beta\).
This means \(\gamma \subset \beta\) and \(\gamma \not = \mathbb{Q}\), so \(\gamma\) satsifies property (I).
To prove (II), pick \(p \in \gamma\) and we can see that \(p \in \alpha_1\) for some \(\alpha_1 \in A\).
If \(q < p\), then \(q \in \alpha_1\) and \(q \in \gamma\) holds, proving (II).
Thus \(\gamma \in \mathbb{R}\) and \(\alpha \leq \gamma\) by the definition of \(\gamma\), so \(\gamma\) is an upper bound of \(A\).

Now, suppose that \(\delta < \gamma\) where \(\delta\) is also an upper bound of \(A\).
Then there exists \(s \in \gamma\) such that \(s \not = \delta\).
Since \(s \in \gamma\), \(s \in \alpha\) for some \(\alpha \in A\), so \(\delta \geq \alpha\) is impossible.
In conclusion, such \(\delta\) cannot exist, so we get the desired result: \(\gamma = \sup A\).

\subsection{Proof for addition axioms}
Define \(0^*\) to be the set of all nonpositive rational numbers.
\(0^*\) is a cut because it satisfies property (I) and (II).

\subsubsection{Proof for (A1)}
For all \(\alpha, \beta \in \mathbb{R}\), the sum of two cuts \(\alpha + \beta\) is a nonempty subset of \(\mathbb{Q}\) by the definition of addition.
Take \(r' \not \in \alpha, s' \not \in \beta\), and \(r' + s' > r + s\) for all \(r \in \alpha, s \in \beta\) holds as \(\mathbb{Q}\) is an ordered field.
Thus \(r' + s' \not \in \alpha + \beta\) and \(\alpha + \beta \not = \mathbb{Q}\), so \(\alpha + \beta\) satisfies property (I).

For \(p \in \alpha + \beta\), then \(p = r + s\) where \(r \in \alpha, s \in \beta\).
If \(q < p\), \(q - s < p - s = r\) holds, so \(q - s \in \alpha\) and \(q = (q - s) + s \in \alpha + \beta\).
Thus property (II) holds and \(\alpha + \beta\) is a cut.

\subsubsection{Proof for (A2)}
By definition, for all \(\alpha, \beta \in \mathbb{R}\), \(\alpha + \beta\) is the set of all \(r + s\) where \(r \in \alpha, s \in \beta\).
Similarly, \(\beta + \alpha\) is the set of all \(s + r\), and \(r + s = s + r\) for all \(r \in \alpha, s \in \beta\)  as \(\mathbb{Q}\) is a field.
So \(\alpha + \beta = \beta + \alpha\) holds.

\subsubsection{Proof for (A3)}
Similar to the proof for (A2), for all \(\alpha, \beta, \gamma \in \mathbb{R}\), \((\alpha + \beta) + \gamma\) is the set of all \((r + s) + t\) where \(r \in \alpha, s \in \beta, t \in \gamma\).
Likewise, \(\alpha + (\beta + \gamma)\) is the set of all \(r + (s + t)\), and \(r + (s + t) = (r + s) + t\) for all \(r \in \alpha, s \in \beta, t \in \gamma\) as \(\mathbb{Q}\) is a field.
Thus, \((\alpha + \beta) + \gamma = \alpha + (\beta + \gamma)\) holds.

\subsubsection{Proof for (A4)}
For \(r \in \alpha\) and \(s \in 0^*\), \(r + s \leq r\), so \(r + s \in \alpha\) and \(\alpha + 0^* \subset \alpha\) holds.

For all \(p, r \in \alpha\) where \(r \geq p\), \(p - r \in 0^*\) and \(p = r + (p - r) \in \alpha + 0^*\) holds.
Thus \(\alpha \subset \alpha + 0^*\) and we obtain the desired result: \(\alpha + 0^* = \alpha\).

\subsubsection{Proof for failure of (A5)}
Suppose that (A5) holds in this particular construction of \(\mathbb{R}\).
Let \(\alpha\) be a set of negative rationals, then \(\alpha\) is a cut by definition, and \(\beta \in \mathbb{R}\) such that \(\alpha + \beta = 0^*\) exists.
From \(0 \in 0^* \subset \alpha + \beta\), \(0 \in \alpha + \beta\) holds.
By the definition of addition, there exists \(r \in \alpha, s \in \beta\) such that \(r + s = 0\).
Since \(r < 0\) for all \(r \in \alpha\), there exists \(s' \in \beta\) such that \(s' > 0\).
However, \(-s' / 2 \in \alpha\) by definition, and \(-s' / 2 + s' = s' / 2 \in \alpha + \beta = 0^*\), but it is a contradiction because \(s' / 2 > 0\).
Thus, this particular construction of \(\mathbb{R}\) without property (III) of cuts cannot satisfy the axiom (A5).

\end{document}
