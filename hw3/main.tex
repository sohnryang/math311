\documentclass{scrartcl}
\usepackage[margin=3cm]{geometry}
\usepackage{amsmath}
\usepackage{amssymb}
\usepackage{amsthm}
\usepackage{blindtext}
\usepackage{datetime}
\usepackage{fontspec}
\usepackage{graphicx}
\usepackage{kotex}
\usepackage{mathrsfs}
\usepackage{mathtools}
\usepackage{pgf,tikz,pgfplots}

\pgfplotsset{compat=1.15}
\usetikzlibrary{arrows}
\newcommand\Overline[2][0.8pt]{%
  \begin{tikzpicture}[baseline=(a.base)]
    \node[inner xsep=0pt,inner ysep=1.5pt] (a) {$#2$};
    \draw[line width= #1] (a.north west) -- (a.north east);
  \end{tikzpicture}
}
\newtheorem{theorem}{Theorem}

\setmainhangulfont{Noto Serif CJK KR}[
  UprightFont=* Light, BoldFont=* Bold,
  Script=Hangul, Language=Korean, AutoFakeSlant,
]
\setsanshangulfont{Noto Sans CJK KR}[
  UprightFont=* DemiLight, BoldFont=* Medium,
  Script=Hangul, Language=Korean
]
\setmathhangulfont{Noto Sans CJK KR}[
  SizeFeatures={
    {Size=-6,  Font=* Medium},
    {Size=6-9, Font=*},
    {Size=9-,  Font=* DemiLight},
  },
  Script=Hangul, Language=Korean
]
\title{MATH311: Homework 3 (due Mar. 14)}
\author{손량(20220323)}
\date{Last compiled on: \today, \currenttime}

\newcommand{\un}[1]{\ensuremath{\ \mathrm{#1}}}

\begin{document}
\maketitle

\section{Section 1 \#9}
For all \(z, w\) in \(\mathbb{C}\), if \(z = w\) then \(z > w\) or \(z < w\) is impossible because \(a < c\), \(a > c\), \(b < d\), \(b > d\) are all false as the set of reals is an ordered set.
Likewise, if \(z < w\) then \(z = w\) is impossible because if \(a < c\), then \(a = c\) is false by orderedness of the set of reals.
If \(a = c\) but \(b < d\), then \(b = d\) is false by similar reason.
\(z > w\) is also impossible because \(a > c\) is false as one of \(a < c\) or \(a = c\) is true.
By similar argument, it can be proved that \(z = w\) and \(z < w\) is false if \(z > w\).

Let \(z = a + bi, w = c + di, v = e + fi\) where \(z < w\), and \(w < v\).
By the definition of the lexicographic order, \(a \leq c\) and \(c \leq e\) holds.
If \(a = c = e\), then \(b < d\) and \(d < f\) holds by the definition of the order, and \(b < f\) is true since the set of reals is an ordered set, so \(z < v\) holds by the definition.
Otherwise, \(a < e\) holds since the set of reals is an ordered set, and \(z < v\) also holds by the definition.

Let \(A\) be a set of complex numbers \(z = x + yi\) where \(x < 0\).
Since \(x < 0\), \(z < 0\) holds for all \(z \in A\), by the definition of lexicographic order.
Thus, \(A\) is bounded above since it has an upper bound.
Let \(L\) be a set of upper bounds of \(A\).
For all \(B = a + bi\) in \(L\), \(B \geq z\) for all \(z\) in \(A\), thus \(a \geq x\) should hold where \(z = x + yi\) by definition.
If \(a < 0\), then \(a / 2 \in A\), and \(a / 2 > a + bi\) by the definition of lexicographic order.
Thus, \(a \geq 0\) should hold for all \(a + bi \in L\).
On the other hand, for all \(a' + b'i\) where \(a' \geq 0\), \(a' + b'i > z\) by the definition of lexicographic order.
This implies that \(L\) is a set of complex numbers with nonnegative real part.
Suppose that there exists a minimum element \(u + vi\) in \(L\).
Since \(0 \in L\), \(u \leq 0\) holds because \(u + vi > 0\) is false, thus \(u > 0\) is also false.
We know that \(u\) is nonnegative, so \(u = 0\).
However, \(u + vi\) for all \(v \in \mathbb{R}\) cannot be the least element of \(L\) since \(u + (v - 1)i < u + vi\), which is a contradiction.
Therefore, such \(u + vi\) cannot exist and \(A\) does not have the least upper bound.
Since a subset of our ordered set, \(A\) does not have the least upper bound, this ordered set does not have the least-upper-bound property.

\section{Section 1 \#15}
Let \(A = \sum |a_j|^2, B = \sum |b_j|^2, C = \sum a_j \Overline{b_j}\).
Then we can write the following:
\[
  \sum_{j=1}^n \left| a_j \right|^2 \sum_{j=1}^n \left| b_j \right|^2 - \left| \sum_{j=1}^n a_j \Overline{b_j} \right|^2 = AB - |C|^2 \geq 0
\]

If \(b_1 = \dots = b_n = 0\), then \(B = 0\) and the equality holds.
Assume that \(B > 0\), by theorem~1.31 we can write
\begin{align*}
  AB - |C|^2 &= \frac{B(AB - |C|^2)}{B} = \frac{B^2 A - B|C|^2}{B} \\
             &= \frac{1}{B} \left( B^2 \sum_{j=1}^n |a_j|^2 - B\Overline{C} \sum_{j=1}^n a_j \Overline{b_j} - BC \sum_{j=1}^n \Overline{a_j}b_j + |C|^2 \sum_{j=1}^n |b_j|^2 \right) \\
             &= \frac{1}{B} \sum_{j=1}^n (Ba_j - Cb_j)(B\Overline{a_j} - \Overline{Cb_j}) = \frac{1}{B} \sum_{j=1}^n |Ba_j - Cb_j|^2
\end{align*}

If the equality holds, then \(|Ba_j - Cb_j|^2 = 0\) for \(j = 1, 2, \dots, n\), thus \(Ba_j - Cb_j = 0\) by theorem~1.33.
In other words, there exists some ratio \(r \in \mathbb{C}\) where \(a_j - rb_j = 0\) if the equality holds.
We can prove the converse: suppose that there exists some \(r \in \mathbb{C}\) where \(a_j - rb_j = 0\) for \(j = 1, 2, \dots, n\).
Then we can write
\begin{align*}
  A &= \sum_{j=1}^n |a_j|^2 = \sum_{j=1}^n |rb_j|^2 = \sum_{j=1}^n r^2 |b_j|^2 = r^2 \sum_{j=1}^n |b_j|^2 = r^2 B \\
  C &= \sum_{j=1}^n a_j \Overline{b_j} = \sum_{j=1}^n rb_j\Overline{b_j} = r\sum_{j=1}^n |b_j|^2 = rB
\end{align*}
and \(AB - |C|^2 = r^2 B^2 - |rB|^2 = 0\) by theorem~1.33, so the equality holds.

In conclusion, the equality holds in the Schwarz inequality iff \(b_1 = \dots = b_n =0\) or there exists some \(r \in \mathbb{C}\) such that \(a_j = rb_j\) for \(j = 1, 2, \dots, n\).

\section{Section 1 \#17}
By the definition of norm, we can write
\begin{align*}
  |\mathbf{x} + \mathbf{y}|^2 + |\mathbf{x} - \mathbf{y}|^2 &= (\mathbf{x} + \mathbf{y}) \cdot (\mathbf{x} + \mathbf{y}) + (\mathbf{x} - \mathbf{y}) \cdot (\mathbf{x} - \mathbf{y}) \\
                                                            &= |\mathbf{x}|^2 + |\mathbf{y}|^2 + 2\mathbf{x} \cdot \mathbf{y} + |\mathbf{x}|^2 + |\mathbf{y}|^2 - 2\mathbf{x} \cdot \mathbf{y} \\
                                                            &= 2|\mathbf{x}|^2 + 2|\mathbf{y}|^2
\end{align*}

Consider \(\mathbf{x}, \mathbf{y}\) as the adjacent sides of a parallelogram.
Then \(\mathbf{x + y}, \mathbf{x - y}\) can be considered as two diagonals of the parallelogram.
From the equation, we can know that the sum of squares of the length of sides of parallelogram is equal to the sum of square of the length of diagonals.

\section{Section 2 \#2}
Let \(A_m\) be a set of equations \(a_0 z^n + a_1 z^{n - 1} + \dots + a_{n - 1} z + a_n = 0\), where \(a_0, a_1, \dots, a_n\) are integers and not all zero, and \(n + |a_0| + |a_1| + \dots + |a_n| = m\).
For all \(a_0 z^n + a_1 z^{n - 1} + \dots + a_{n - 1} z + a_n = 0\), there exists a set \(A_m\) which has \(a_0 z^n + a_1 z^{n - 1} + \dots + a_{n - 1} z + a_n = 0\) as an element.
Let \(B_m\) be a set of solutions of equations in \(A_m\).
Since \(A_m\) is a finite set, and every equation in \(A_m\) has finite number of solutions by the fundamental theorem of algebra, \(B_m\) is also a finite set.
So the union \(\bigcup_{i = 1}^\infty B_i\) is a countable set by the theorem~2.12.

\section{Section 2 \#9}
\subsection{Proof for (a)}
Suppose that there exists a point \(q \in E^\circ\) such that all of its neighborhood contains a point \(r\) which is not in \(E^\circ\).
Since \(q\) is an interrior point of \(E\), there exists a neighborhood \(N\) of \(q\) such that \(N \subset E\).
By the assumption, there exists a point \(r \in N\) such that \(r \not \in E^\circ\).
By the theorem~2.19, \(N\) is an open set so \(r\) is an interrior point of \(N\).
Then, there exists a neighborhood \(M\) of \(r\) such that \(M \subset N\).
Since \(N \subset E\), \(M \subset E\) holds and \(r\) is also an interrior point of \(E\), which is a contradiction.
Thus, such \(q\) cannot exist, so for all \(p \in E^\circ\), there exists a neighborhood \(N\) of \(p\) such that \(N \subset E^\circ\), which means that \(p\) is an interrior point.
In conclusion, every point of \(E^\circ\) is an interrior point of \(E^\circ\), so \(E^\circ\) is an open set by definition.

\subsection{Proof for (b)}
By definition, \(E\) is open iff its points is an interrior point, so it is same as saying that \(E = E^\circ\).

\subsection{Proof for (c)}
Suppose that there exist a point \(q \in G\) such that \(q \not \in E^\circ\).
Since \(G\) is an open set, \(q\) is an interrior point of \(G\), so there exists a neighborhood \(N\) of \(q\) such that \(N \subset G\).
However, \(G \subset E\) implies that \(N \subset E\), thus \(q\) is an interrior point of \(E\) by definition, so \(q \in E^\circ\) which is a contradiction.
In conclusion, such \(q\) does not exist, and for all \(p \in G\), \(p \in E^\circ\), so \(G \subset E^\circ\).

\section{Section 2 \#10}
\begin{theorem}
  All subset of \(X\) are open set.
\end{theorem}
\begin{proof}
  Let \(A\) be a subset of \(X\).
  For all \(p \in A\), consider a neighborhood \(N = \{q \in A;\; d(p, q) < 1\}\).
  Since \(d(p, q)\) is 0 or 1 for all \(p, q \in X\), \(d(p, q) < 1\) is equivalent to \(d(p, q) = 0\).
  Thus, \(N = \{q \in A | p = q\}\), so \(N \subset A\) since \(p \in A\) and \(p\) is interrior point of \(A\).
  In conclusion, since every point in \(A\) is interrior point, \(A\) is an open set.
\end{proof}

\begin{theorem}
  All subset of \(X\) are closed set.
\end{theorem}
\begin{proof}
  Let \(A\) be as subset of \(X\).
  As we proved earlier, \(A^C \subset X\) is open.
  By the collorary of theorem~2.23, \(A\) is closed since its complement is open.
\end{proof}

\section{Section 2 \#11}
\begin{theorem}
  \(d_1(x, y) = (x - y)^2\) is not a metric.
\end{theorem}
\begin{proof}
  For all \(x, y, z \in \mathbb{R}\), \(d_1(x, z) + d_1(z, y) - d_1(x, y)\) can be written as
  \begin{align*}
    d_1(x, z) + d_1(z, y) - d_1(x, y) &= (x - z)^2 + (z - y)^2 - (x - y)^2 \\
                                      &= x^2 - 2xz + z^2 + z^2 - 2zy + y^2 - (x^2 - 2xy + y^2) \\
                                      &= 2z^2 - 2xz - 2zy + 2xy = 2(z - x)(z - y)
  \end{align*}
  If \(x < z < y\) then \(d_1(x, z) + d_1(z, u) - d_1(x, y) = 2(z - x)(z - y) < 0\), so \(d_1\) does not satisfy the condition~(c) of definition~2.15.
  Thus, \(d_1\) is not a metric.
\end{proof}

\begin{theorem}
  \(d_2(x, y) = \sqrt{|x - y|}\) is a metric.
\end{theorem}
\begin{proof}
  For all \(x, y, z \in \mathbb{R}\), \(d_2(x, x) = \sqrt{|x - x|} = 0\) and \(d_2(x, y) = \sqrt{|x - y|} = \sqrt{|y - x|} = d_2(y, x)\) holds.
  If \(x \not = y\), then \(d_2(x, y) = \sqrt{|x - y|}\) and \(|x - y| > 0\), so \(\sqrt{|x - y|} > 0\).
  \((d_2(x, z) + d_2(z, y))^2 - (d_2(x, y))^2\) can be written as
  \begin{align*}
    (d_2(x, z) + d_2(z, y))^2 - (d_2(x, y))^2 &= \left(\sqrt{|x - z|} + \sqrt{|z - y|}\right)^2 - |x - y| \\
                                              &= |x - z| + |z - y| + 2\sqrt{|x - z||z - y|} - |x - y| \\
  \end{align*}
  Since \(|x - z| + |z - y| \geq |x - y|\) by theorem~1.37 and \(\sqrt{|x - z||z - y|} \geq 0\), \((d_2(x, z) + d_2(z, y))^2 - (d_2(x, y))^2 \geq 0\) holds.
  From \(d_2(x, y) \geq 0\), we can conclude that \(d_2(x, z) + d_2(z, y) \geq d_2(x, y)\), thus \(d_2\) is a metric.
\end{proof}

\begin{theorem}
  \(d_3(x, y) = |x^2 - y^2|\) is not a metric.
\end{theorem}
\begin{proof}
  Let \(x\) be a nonzero real number.
  Since \(2x \not = 0\), we can say that \(x \not = -x\).
  However, \(d_3(x, -x) = |x^2 - (-x)^2| = |x^2 - x^2| = 0\) holds, so \(d_3\) does not satisfy the condition~(a) of definition~2.15.
  Thus, \(d_3\) is not a metric.
\end{proof}

\begin{theorem}
  \(d_4(x, y) = |x - 2y|\) is not a metric.
\end{theorem}
\begin{proof}
  Let \(x\) be a nonzero real number.
  \(d_4(x, x) = |x - 2x| = |-x| = |x| \not = 0\) holds, so \(d_4\) does not satisfy the condition~(a) of definition~2.15.
  Thus, \(d_4\) is not a metric.
\end{proof}

\begin{theorem}
  \(d_5(x, y) = |x - y| / (1 + |x - y|)\) is a metric.
\end{theorem}
\begin{proof}
  For all \(x, y, z \in \mathbb{R}\), \(d_5(x, x) = |x - x| / (1 + |x - x|) = 0\) and \(d_5(x, y) = |x - y| / (1 + |x - y|) = |y - x| / (1 + |y - x|) = d_5(y, x)\) holds.
  If \(x \not = y\), then \(d_5(x, y) = |x - y| / (1 + |x - y|)\) and \(|x - y| > 0\), so \(|x - y| / (1 + |x - y|) > 0\).
  We can write
  \begin{align*}
    d_5(x, z) + d_5(z, y) &= \frac{|x - z|}{1 + |x - z|} + \frac{|z - y|}{1 + |z - y|} \\
                          &= \frac{|x - z|(1 + |z - y|) + |z - y|(1 + |x - z|)}{(1 + |x - z|)(1 + |z - y|)} \\
                          &= \frac{|x - z| + |z - y| + 2|x - z||z - y|}{1 + |x - z| + |z - y| + |x - z||z - y|} \\
                          &= 1 - \frac{1 - |x - z||z - y|}{1 + |x - z| + |z - y| + |x - z||z - y|}
  \end{align*}
  We can also write
  \begin{align*}
    1 - d_5(x, y) &= \frac{1}{1 + |x - y|} \geq \frac{1}{1 + |x - z| + |z - y|} \geq \frac{1}{1 + |x - z| + |z - y| + |x - z||z - y|} \\
                  &\geq \frac{1 - |x - z||z - y|}{1 + |x - z| + |z - y| + |x - z||z - y|} = 1 - (d_5(x, z) + d_5(z, y))
  \end{align*}
  Thus, \(d_5(x, z) + d_5(z, y) \geq d_5(x, y)\) holds and we can conclude that \(d_5\) is a metric.
\end{proof}

\end{document}
