\documentclass{scrartcl}
\usepackage[margin=3cm]{geometry}
\usepackage{amsmath}
\usepackage{amssymb}
\usepackage{amsthm}
\usepackage{blindtext}
\usepackage{datetime}
\usepackage{fontspec}
\usepackage{graphicx}
\usepackage{kotex}
\usepackage{mathrsfs}
\usepackage{mathtools}
\usepackage{pgf,tikz,pgfplots}

\pgfplotsset{compat=1.15}
\usetikzlibrary{arrows}

\newcommand\Overline[2][0.8pt]{%
  \begin{tikzpicture}[baseline=(a.base)]
    \node[inner xsep=0pt,inner ysep=1.5pt] (a) {$#2$};
    \draw[line width= #1] (a.north west) -- (a.north east);
  \end{tikzpicture}
}
\newtheorem{theorem}{Theorem}

\setmainhangulfont{Noto Serif CJK KR}[
  UprightFont=* Light, BoldFont=* Bold,
  Script=Hangul, Language=Korean, AutoFakeSlant,
]
\setsanshangulfont{Noto Sans CJK KR}[
  UprightFont=* DemiLight, BoldFont=* Medium,
  Script=Hangul, Language=Korean
]
\setmathhangulfont{Noto Sans CJK KR}[
  SizeFeatures={
    {Size=-6,  Font=* Medium},
    {Size=6-9, Font=*},
    {Size=9-,  Font=* DemiLight},
  },
  Script=Hangul, Language=Korean
]
\title{MATH311: Homework 11 (due May. 23)}
\author{손량(20220323)}
\date{Last compiled on: \today, \currenttime}

\newcommand{\un}[1]{\ensuremath{\ \mathrm{#1}}}
\newcommand{\imag}{\operatorname{Im}}
\newcommand{\real}{\operatorname{Re}}
\newcommand{\Log}{\operatorname{Log}}
\newcommand{\Arg}{\operatorname{Arg}}
\DeclareMathOperator*{\Res}{Res}

\begin{document}
\maketitle

\section{Section 6 \#5}
Let \(f : [a, b] \to \mathbb{R}\) as
\begin{align*}
  f(x) = \begin{cases}
    1 & (x \in \mathbb{Q} \cap [a, b]) \\
    -1 & (x \in (\mathbb{R} \setminus \mathbb{Q}) \cap [a, b])
  \end{cases}
\end{align*}
Then, \(|f(x)| \leq 1\) for all \(x \in [a, b]\) so \(f\) is bounded and \((f(x))^2 = 1\) for all \(x \in [a, b]\), thus \(f^2 \in \mathscr{R}\).
However, for all partition \(P = \{p_0, \dots, p_n\}\) of \([a, b]\), by Archimedean property there exists some rational number in \([p_{i - 1}, p_i]\) for \(i = 1, 2, \dots, n\).
Thus, \(U(P, f) = b - a\).
On the other hand, by Archimedean property there exists some rational number \(q_i\) in \((p_{i - 1} / \sqrt{2}, p_i / \sqrt{2})\) for \(i = 1, 2, \dots, n\), so there exists some irrational number \(q_i \sqrt{2} \in (p_{i - 1}, p_i)\) for \(i = 1, 2, \dots, n\).
Thus, \(L(P, f) = a - b\).
For all partition \(P\) of \([a, b]\), \(U(P, f) - L(P, f) = 2(b - a)\) so \(f \not \in \mathscr{R}\).

Let \(\phi : \mathbb{R} \to \mathbb{R}\) as
\begin{align*}
  \phi(x) = \begin{cases}
    \sqrt[3]{x} & (x \geq 0) \\
    -\sqrt[3]{-x} & (x < 0)
  \end{cases}
\end{align*}
and \(\phi\) is continuous on \(\mathbb{R}\).
Then, by theorem~6.11 \(f = \phi(f^3)\) is integrable on \([a, b]\).

\section{Section 6 \#6}
Recall the construction of Cantor's set we made earlier.
By construction, \(P \subset E_n\) where \(E_n\) consists of \(2^n\) closed intervals \([u_{n, i}, v_{n, i}] \quad (i = 1, 2, \dots, 2^n)\), and \(v_{n, i} - u_{n, i} = 3^{-n}\).
Also, the definition of \(E_n\) implies \(u_{n, i + 1} - v_{n, i} \geq 3^{-n - 1} \quad (i = 1, 2, \dots, 2^n - 1)\) as \(E_n\) is created by removing middle thirds of intervals in \(E_{n - 1}\).
Let \(E^*_n = \cup^{2^n}_{i = 1} (u^*_{n, i}, v^*_{n, i})\) where \(u^*_{n, i} = u_{n, i} - 3^{-n - 1},\, v^*_{n, i} = v_{n, i} + 3^{-n - 1}\).
Then, \(u^*_{n, i + 1} - v^*_{n, i} = u_{n, i + 1} - v_{n, i} - 2 \cdot 3^{-n - 1} \geq 3^{-n - 2}\) so \((u^*_{n, i}, v^*_{n, i})\) are disjoint and \(E_n \subset E^*_n\).
Let \(K = [0, 1] - E^*_k\), then \(K\) is compact since it is closed and bounded in \(\mathbb{R}\).
Fix \(\epsilon > 0\).
Since \(f\) is continuous on \([0, 1] - P\), so it is continuous on \(K\), which implies uniform continuity on \(K\).
Thus, there exists \(\delta > 0\) such that \(|f(s) - f(t)| < \epsilon\) if \(s, t \in K\) and \(|s - t| < \delta\).
Let \(Q = \{x_0, \dots, x_n\}\) be a partition of \([0, 1]\), which satisfies the following conditions:
\begin{enumerate}
  \item Each \(u^*_{k, i}\) and \(v^*_{k, i}\) which is in \([0, 1]\) occurs in \(Q\).
  \item No point in \(E^*_k\) does not occur in \(Q\).
  \item If \(x_{i - 1}\) is not one of the \(u^*_{k, j}\) and zero, \(\Delta x_i < \delta\).
\end{enumerate}
Let \(M = \sup |f(x)|\).
Let \(A = \{i \in \{1, 2, \dots, n\};\; x_{i - 1} \in \{0, u^*_{k, 1}, \dots, u^*_{k, 2^k}\}\}\) and \(B = \{1, 2, \dots, n\} \setminus A\).
For all \(i = 1, 2, \dots, n\), \(M_i - m_i \leq 2M\).
If \(i \in A\), \(\Delta x_i < \delta\) so \(M_i - m_i \leq \epsilon\).
If \(i \in B\), \(x_i \in \{v^*_{k, 1}, \dots, v^*_{k, 2^k}, 1\}\) so \(\Delta x_i \leq 3^{-k} + 2 \cdot 3^{-k - 1}\).
Then we can write
\begin{align*}
  U(Q, f) - L(Q, f)
  &= \sum^n_{i = 1} (M_i - m_i) \Delta x_i
  = \sum_{i \in A} (M_i - m_i) \Delta x_i + \sum_{i \in B} (M_i - m_i) \Delta x_i \\
  &\leq \sum_{i \in A} \epsilon \Delta x_i + \sum_{i \in B} 2M \cdot \frac{5}{3^{k + 1}}
  \leq \epsilon \sum^n_{i = 1} \Delta x_i + 2^k \cdot 2M \cdot \frac{5}{3^{k + 1}}
  < \epsilon + 5M\epsilon
\end{align*}
Since the choice of \(\epsilon\) is arbitrary, \(U(Q, f) - L(Q, f)\) can be made small as we like, and \(f \in \mathscr{R}\) on \([0, 1]\).

\section{Section 6 \#7}
\subsection{Proof for (a)}
We can write
\begin{align}\label{sec3_1_ineq}
  L(\{0, c\}, f) \leq \int^c_0 f(x) dx \leq U(\{0, c\}, f)
\end{align}
Since \(f \in \mathscr{R}\) on \([0, 1]\), \(f\) is bounded on \([0, 1]\) and there exists some \(M > 0\) such that \(\sup_{x \in [0, c]} |f(x)| \leq M\).
Then we can write (\ref{sec3_1_ineq}) as
\begin{align*}
  -cM \leq c \inf_{x \in [0, c]} f(x) \leq \int^c_0 f(x) dx \leq c \sup_{x \in [0, c]} f(x) \leq cM
\end{align*}
by sandwich theorem, we know that \(\lim_{c \to 0} \int^c_0 f(x) dx = 0\), so we can write
\begin{align*}
  \lim_{c \to 0} \int^1_c f(x) dx
  &= \lim_{c \to 0} \left( \int^1_0 f(x) dx - \int^c_0 f(x) dx \right)
  = \int^1_0 f(x) dx - \lim_{c \to 0} \int^c_0 f(x) dx \\
  &= \int^1_0 f(x) dx
\end{align*}

\subsection{Solution for (b)}
Let \(f(x) = (-1)^n (n + 1)\), where \(n\) is the maximum integer \(n \leq 1 / x\) then \(f \in \mathscr{R}\) on \([c, 1]\) for all \(c > 0\).
If \(N \leq 1 / c \leq N + 1\) for \(c > 0\), we can write
\begin{align*}
  \int^1_c f(x) dx
  &= (-1)^N (N + 1) \left( \frac{1}{N} - c \right) + \sum^{N - 1}_{k = 1} (-1)^k (k + 1) \left( \frac{1}{k} - \frac{1}{k + 1} \right) \\
  &= (-1)^N (N + 1) \left( \frac{1}{N} - c \right) + \sum^{N - 1}_{k = 1} \frac{(-1)^k}{k} \\
\end{align*}
and
\begin{align*}
  0 \leq (N + 1) \left( \frac{1}{N} - c \right) \leq \frac{1}{N}
\end{align*}
since \(N \to \infty\) as \(c \to 0\), the first term converges to zero and by alternating series theorem \(\sum^{N - 1}_{k = 1} (-1)^k / k\) converges as \(N \to \infty\), so the integral converges.
However, we can write
\begin{align*}
\int^1_c |f(x)| dx
= (N + 1) \left( \frac{1}{N} - c \right) + \sum^{N - 1}_{k = 1} \frac{1}{k}
\end{align*}
and
\begin{align*}
  \sum^{N - 1}_{k = 1} \frac{1}{k} \leq \int^1_c |f(x)| dx \leq \sum^N_{k = 1} \frac{1}{k}
\end{align*}
since harmonic series diverges, the integral also diverges as \(c \to 0\).

\section{Section 6 \#8}
Suppose that \(\sum^\infty_{n = 1} f(n)\) converges.
Let \(P = \{x_0, \dots, x_n\} = \{1, 2, \dots, n + 1\}\).
Then we can write
\begin{align*}
  U(P, f)
  &= \sum^n_{i = 1} M_i \Delta x_i
  = \sum^n_{i = 1} f(x_{i - 1}) \Delta x_i
  = \sum^n_{i = 1} f(i) \\
  L(P, f)
  &= \sum^n_{i = 1} m_i \Delta x_i
  = \sum^n_{i = 1} f(x_i) \Delta x_i
  = \sum^{n + 1}_{i = 2} f(i)
\end{align*}
and
\begin{align}\label{sec4_ineq}
  L(P, f)
  = \sum^{n + 1}_{i = 2} f(i)
  \leq \int^{n + 1}_1 f(x) dx
  \leq \sum^n_{i = 1} f(i)
  = U(P, f)
\end{align}
Using sandwich theorem,
\begin{align*}
  \lim_{n \to \infty} \sum^{n + 1}_{i = 2} f(i)
  \leq \lim_{n \to \infty} \int^{n + 1}_1 f(x) dx
  \leq \lim_{n \to \infty} \sum^n_{i = 1} f(i)
\end{align*}
implies the existence \(\int^\infty_1 f(x) dx\).

Suppose that \(\int^\infty_1 f(x) dx\) converges.
From (\ref{sec4_ineq}), we obtain
\begin{align*}
  \sum^n_{i = 1} f(i) \leq f(1) + \int^n_1 f(x) dx \leq f(1) + \int^\infty_1 f(x) dx
\end{align*}
as \(f\) is nonnegative, so \(\int^n_1 f(x) dx\) is monotonically increasing.
Since \(\sum^n_{k = 1} f(k)\) is a monotonically increasing sequence which is bounded above, it converges by monotone convergence theorem.

\section{Section 6 \#10}
\subsection{Proof for (a)}
Fix \(v\).
Let \(\phi(u) = u^p / p + v^q / q - uv\).
Then \(\phi(u) = u^{p - 1} - v\).
For \(0 \leq u < v^{1 / (p - 1)}\), \(\phi'(u) < 0\) so \(\phi(u)\) decreases there.
For \(u > v^{1 / (p - 1)}\), \(\phi'(u) > 0\) so \(\phi(u)\) increases there.
Thus, \(\phi(v^{1 / (p - 1)})\) is the minimum of \(\phi(u)\) and we can write
\begin{align*}
  \phi(v^\frac{1}{p - 1})
  = \frac{v^\frac{p}{p - 1}}{p} + \frac{v^q}{q} - v^\frac{p}{p - 1}
  = \left( \frac{1}{p} + \frac{1}{q} \right) v^q - v^q
  = 0
\end{align*}
for all \(u \geq 0\), \(\phi(u) \geq \phi(v^{1 / (p - 1)}) = 0\) so the inequality holds.
If \(u^p = v^q\), we can write
\begin{align*}
  \frac{u^p}{p} + \frac{v^q}{q} &= \frac{u^p}{p} + \frac{u^p}{q} = u^p \\
  uv &= u \cdot u^{p / q} = u^\frac{p + q}{q} = u^{1 + p / q} = u^{p \left( \frac{1}{p} + \frac{1}{q} \right)} = u^p
\end{align*}
Thus the equality holds.

\subsection{Proof for (b)}
From the inequality we proved in (a) and properties of integral,
\begin{align*}
  \int^b_a fg d\alpha
  \leq \int^b_a \left( \frac{f^p}{p} + \frac{g^q}{q} \right) d\alpha
  = \frac{1}{p} \int^b_a f^p d\alpha + \frac{1}{q} \int^b_a g^q d\alpha
  = \frac{1}{p} + \frac{1}{q} = 1
\end{align*}

\subsection{Proof for (c)}
Let \(A = \int^b_a |f|^p d\alpha,\, B = \int^b_a |g|^q d\alpha\) and \(\hat{f}(x) = f(x) / A^{1 / p},\, \hat{g}(x) = g(x) / B^{1 / q}\).
Assume that \(A > 0\) and \(B > 0\).
Then we can write
\begin{align*}
  \int^b_a |\hat{f}|^p d\alpha
  &= \int^b_a \left| \frac{f}{A^\frac{1}{p}} \right|^p d\alpha
  = \frac{1}{A} \int^b_a |f|^p d\alpha
  = 1 \\
  \int^b_a |\hat{g}|^q d\alpha
  &= \int^b_a \left| \frac{g}{B^\frac{1}{q}} \right|^q d\alpha
  = \frac{1}{B} \int^b_a |g|^q d\alpha
  = 1
\end{align*}
By the result proved in (b),
\begin{align}\label{sec5_3_ineq}
  \int^b_a |\hat{f} \hat{g}| d\alpha \leq 1
\end{align}
Suppose that the following holds:
\begin{align*}
  \int^b_a \hat{f} \hat{g} d\alpha = r_0 e^{i\theta_0}
\end{align*}
then we can write
\begin{align*}
  \int^b_a e^{-i\theta_0} \hat{f} \hat{g} d\alpha
  = \real \int^b_a e^{-i\theta_0} \hat{f} \hat{g} d\alpha
  = \int^b_a \real (e^{-i\theta_0} \hat{f} \hat{g}) d\alpha
  = r_0
\end{align*}
and \(\real (e^{-i\theta_0} \hat{f} \hat{g}) \leq |e^{-i\theta_0} \hat{f} \hat{g}| = |\hat{f} \hat{g}|\), so we can write
\begin{align*}
  \left| \int^b_a \hat{f} \hat{g} d\alpha \right| \leq \int^b_a |\hat{f} \hat{g}| d\alpha
\end{align*}
By (\ref{sec5_3_ineq}),
\begin{align*}
  \left| \int^b_a \hat{f} \hat{g} d\alpha \right|
  &\leq \int^b_a |\hat{f} \hat{g}| d\alpha
  \leq 1
  = \left( \int^b_a |\hat{f}|^p d\alpha \right)^\frac{1}{p} \left( \int^b_a |\hat{g}|^q d\alpha \right)^\frac{1}{q} \\
  &= \frac{1}{A^{1 / p} B^{1 / q}} \left( \int^b_a |f|^p d\alpha \right)^\frac{1}{p} \left( \int^b_a |g|^q d\alpha \right)^\frac{1}{q}
\end{align*}
Multiplying both sides with \(A^{1 / p} B^{1 / q}\) gives
\begin{align*}
  \left| \int^b_a fg d\alpha \right| \leq \left( \int^b_a |f|^p d\alpha \right)^\frac{1}{p} \left( \int^b_a |g|^q d\alpha \right)^\frac{1}{q}
\end{align*}
and we get the desired result.

\subsection{Proof for (d)}
Suppose that \(f, g \in \mathscr{R}\) on \([c, 1]\) for every \(c > 0\).
Then we have
\begin{align*}
  \left| \int^1_c fg dx \right| \leq \left( \int^1_c |f|^p dx \right)^\frac{1}{p} \left( \int^1_c |g|^q dx \right)^\frac{1}{q}
\end{align*}
Since \(|f| \geq 0\) and \(|g| \geq 0\), the right hand side increases as \(c \to 0\) and we can write
\begin{align*}
  \left| \int^1_c fg dx \right| \leq \left( \int^1_0 |f|^p dx \right)^\frac{1}{p} \left( \int^1_0 |g|^q dx \right)^\frac{1}{q}
\end{align*}
Since \(|z|\) is continuous for all \(z \in \mathbb{C}\) and by properties of limits, we can write
\begin{align*}
  \lim_{c \to 0} \left| \int^1_c fg dx \right|
  = \left| \lim_{c \to 0} \int^1_c fg dx \right|
  = \left| \int^1_0 fg dx \right|
  \leq \left( \int^1_0 |f|^p dx \right)^\frac{1}{p} \left( \int^1_0 |g|^q dx \right)^\frac{1}{q}
\end{align*}

Now, suppose that \(f, g \in \mathscr{R}\) on \([a, b]\) for every \(b > a\) where \(a\) is fixed.
Then we have
\begin{align*}
  \left| \int^b_a fg dx \right| \leq \left( \int^b_a |f|^p dx \right)^\frac{1}{p} \left( \int^b_a |g|^q dx \right)^\frac{1}{q}
\end{align*}
Since \(|f| \geq 0\) and \(|g| \geq 0\), the right hand side increases as \(b \to \infty\) and we can write
\begin{align*}
  \left| \int^b_a fg dx \right| \leq \left( \int^\infty_a |f|^p dx \right)^\frac{1}{p} \left( \int^\infty_a |g|^q dx \right)^\frac{1}{q}
\end{align*}
Since \(|z|\) is continuous for all \(z \in \mathbb{C}\) and by properties of limits, we can write
\begin{align*}
  \lim_{b \to \infty} \left| \int^b_a fg dx \right|
  = \left| \lim_{b \to \infty} \int^b_a fg dx \right|
  = \left| \int^\infty_a fg dx \right|
  \leq \left( \int^\infty_a |f|^p dx \right)^\frac{1}{p} \left( \int^\infty_a |g|^q dx \right)^\frac{1}{q}
\end{align*}
and we get the desired result.

\section{Section 6 \#11}
By Schwarz inequality,
\begin{align*}
  \left| \int^b_a |f - g||g - h| d\alpha \right|
  &\leq \left( \int^b_a |f - g|^2 d\alpha \right)^\frac{1}{2} \left( \int^b_a |g - h|^2 d\alpha \right)^\frac{1}{2} \\
  \int^b_a (|f - g|^2 + 2|f - g||g - h| + |g - h|^2) d\alpha
  &\leq \left[ \left( \int^b_a |f - g|^2 d\alpha \right)^\frac{1}{2} + \left( \int^b_a |g - h|^2 d\alpha \right)^\frac{1}{2} \right]^2 \\
  \int^b_a (|f - g| + |g - h|)^2 d\alpha
  &\leq \left[ \left( \int^b_a |f - g|^2 d\alpha \right)^\frac{1}{2} + \left( \int^b_a |g - h|^2 d\alpha \right)^\frac{1}{2} \right]^2
\end{align*}
By triangular inequality, \(|f - h| \leq |f - g| + |g - h|\) and since \(|f - g|,\, |g - h|, |f - h|\) are all nonnegative,
\begin{align*}
  \int^b_a |f - h|^2 d\alpha
  &\leq \left[ \left( \int^b_a |f - g|^2 d\alpha \right)^\frac{1}{2} + \left( \int^b_a |g - h|^2 d\alpha \right)^\frac{1}{2} \right]^2 \\
  \left( \int^b_a |f - h|^2 d\alpha \right)^\frac{1}{2}
  &\leq \left( \int^b_a |f - g|^2 d\alpha \right)^\frac{1}{2} + \left( \int^b_a |g - h|^2 d\alpha \right)^\frac{1}{2} \\
  \|f - h\|_2 &\leq \|f - g\|_2 + \|g - h\|_2
\end{align*}
and we obtain the desired result.

\section{Extra Problem}
Fix \(\epsilon > 0\).
There exists an integer \(N > 0\) such that \(1 / (N + 1) < \epsilon / 2\).
Let \(A = \{1, 1/2, 1/3, 2/3, \dots, 1/N, \dots, (N - 1) / N\}\).
For all \(x \in [0, 1]\) such that \(x \not \in A\), \(f(x) \leq 1 / (N + 1) < \epsilon / 2\).
Let \(m\) be the number of elements of \(A\), \(P = \{p_0, p_1, \dots, p_n\}\) which is a partition of \([0, 1]\) where \(n > m\), \(\Delta x_i < \epsilon / 4m\), and \(I_1 = \{i = \{1, \dots, n\};\; A \cap [p_{i - 1}, p_i] \not = \varnothing\}\) and \(I_2 = \{1, \dots, n\} \setminus I_1\).
Then we can write
\begin{align*}
  U(P, f)
  &= \sum^n_{i = 1} M_i \Delta x_i
  = \sum_{i \in I_1} M_i \Delta x_i + \sum_{i \in I_2} M_i \Delta x_i \\
  &\leq \frac{k\epsilon}{4m} + \frac{\epsilon}{2} \sum_{i \in I_2} \Delta x_i
  \leq \frac{k\epsilon}{4m} + \frac{\epsilon}{2}
\end{align*}
where \(k\) is the number of elements in \(I_1\), and there are at most \(2m\) intervals \([p_{i - 1}, p_i]\) such that \([p_{i - 1}, p_i] \cap A\) is nonempty.
Thus \(k \leq 2m\) so \(U(P, f) \leq \epsilon\).
As stated in the solution for \#5, for all \((p_{i - 1}, p_i)\) there exists at least one irrational number in the open interval, so \(L(P, f) = \sum^n_{i = 1} m_i \Delta x_i = 0\).
Then, \(U(P, f) - L(P, f) \leq \epsilon\) and since our choice of \(\epsilon\) is arbitrary, \(f \in \mathscr{R}\) on \([0, 1]\).

\end{document}
