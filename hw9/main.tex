\documentclass{scrartcl}
\usepackage[margin=3cm]{geometry}
\usepackage{amsmath}
\usepackage{amssymb}
\usepackage{amsthm}
\usepackage{blindtext}
\usepackage{datetime}
\usepackage{fontspec}
\usepackage{graphicx}
\usepackage{kotex}
\usepackage{mathrsfs}
\usepackage{mathtools}
\usepackage{pgf,tikz,pgfplots}

\pgfplotsset{compat=1.15}
\usetikzlibrary{arrows}

\newcommand\Overline[2][0.8pt]{%
  \begin{tikzpicture}[baseline=(a.base)]
    \node[inner xsep=0pt,inner ysep=1.5pt] (a) {$#2$};
    \draw[line width= #1] (a.north west) -- (a.north east);
  \end{tikzpicture}
}
\newtheorem{theorem}{Theorem}

\setmainhangulfont{Noto Serif CJK KR}[
  UprightFont=* Light, BoldFont=* Bold,
  Script=Hangul, Language=Korean, AutoFakeSlant,
]
\setsanshangulfont{Noto Sans CJK KR}[
  UprightFont=* DemiLight, BoldFont=* Medium,
  Script=Hangul, Language=Korean
]
\setmathhangulfont{Noto Sans CJK KR}[
  SizeFeatures={
    {Size=-6,  Font=* Medium},
    {Size=6-9, Font=*},
    {Size=9-,  Font=* DemiLight},
  },
  Script=Hangul, Language=Korean
]
\title{MATH311: Homework 9 (due May. 9)}
\author{손량(20220323)}
\date{Last compiled on: \today, \currenttime}

\newcommand{\un}[1]{\ensuremath{\ \mathrm{#1}}}
\newcommand{\imag}{\operatorname{Im}}
\newcommand{\real}{\operatorname{Re}}
\newcommand{\Log}{\operatorname{Log}}
\newcommand{\Arg}{\operatorname{Arg}}
\DeclareMathOperator*{\Res}{Res}

\begin{document}
\maketitle

\section{Section 5 \#1}
For all \(x \in \mathbb{R}\) and \(h > 0\), we can write
\begin{align*}
  |f(x + h) - f(x)| \leq h^2
\end{align*}
Then
\begin{align*}
  \left| \frac{f(x + h) - f(x)}{h} \right| \leq h
\end{align*}
taking the limit \(h \to 0\) to the both sides, by sandwich theorem we know that \(f'(x) = 0\) for all \(x \in \mathbb{R}\).
For all \(x \not = 0\), there exists some \(c \in (\min\{0, x\}, \max\{0, x\})\) such that \(xf'(c) = f(x) - f(0)\) by mean value theorem.
Since \(f'(c) = 0\), \(f(x) = f(0)\) for all \(x \not = 0\), so \(f(x)\) is constant.

\section{Section 5 \#2}
Suppose that \(f\) is not strictly increasing in \((a, b)\).
There exists some \(x_1, x_2\) such that \(x_1 < x_2\) and \(f(x_1) \geq f(x_2)\).
Then, we can write
\begin{align*}
  \frac{f(x_1) - f(x_2)}{x_1 - x_2} \leq 0
\end{align*}
By mean value theorem, there exists some \(c \in (x_1, x_2)\) such that \(f'(c) = (f(x_1) - f(x_2)) / (x_1 - x_2) \leq 0\), which is a contradiction.
Thus, \(f\) is strictly increasing in \((a, b)\).

Fix \(x \in f((a, b))\), and let \(y = f(x)\).
For sufficiently small \(\delta > 0\), there exists some \(x_1, x_2 \in (a, b)\) such that \(f(x_1) = y - \delta\) and \(f(x_2) = y + \delta\) since \(f\) is a strictly increasing function defined on an open set.
Let \(h(k) = g(y + k) - g(y)\) where \(k \in (-\delta, \delta)\).
Since \(f\) is a strictly increasing function, \(g\) is also a strictly increasing function so \(x + h(k) = g(y + k) \in (g(y - \delta), g(y + \delta)) \subset (a, b)\) holds.
Then we can write
\begin{align*}
  \lim_{k \to 0} \frac{g(y + k) - g(y)}{k} = \lim_{k \to 0} \frac{h(k)}{f(x + h(k)) - f(x)}
\end{align*}
Since \(f\) is a strictly increasing function on \((a, b)\), \(f\) maps all open subsets of \((a, b)\) to open set and \(g\) is continuous.
Then \(h(k) \to 0\) as \(k \to 0\), so we can write
\begin{align*}
  \lim_{k \to 0} \frac{h(k)}{f(x + h(k)) - f(x)} = \lim_{h \to 0} \frac{h}{f(x + h) - f(x)} = \frac{1}{f'(x)}
\end{align*}
Let \(h(t) = g(f(t))\), then since \(g\) is differentiable at \(f(x)\), \(h\) is differentiable at \(x\) and \(h'(x) = g'(f(x))f'(x)\).
Since \(h(t) = g(f(t)) = t\), we can write
\begin{align*}
  g'(f(x)) = \frac{1}{f'(x)}
\end{align*}

\section{Section 5 \#5}
Fix \(\epsilon > 0\).
Since \(f'(x) \to 0\) as \(x \to \infty\), there exists \(M \in \mathbb{R}\) such that \(x > M\) implies \(|f'(x)| < \epsilon\).
By mean value theorem, there exists \(z \in (x, x + 1)\) such that \(f'(z) = f(x + 1) - f(x) = g(x)\).
Then, \(x > M\) implies \(|g(x)| = |f(x + 1) - f(x)| = |f'(z)| < \epsilon\) as \(z > M\) so \(g(x) \to 0\) as \(x \to \infty\).

\section{Section 5 \#6}
By mean value theorem, there exists \(z \in (0, x)\) such that \(f'(z) = f(x) / x = (f(x) - f(0)) / (x - 0)\) for all \(x > 0\).
Since \(f'\) is monotonically increasing, \(f'(x) \geq f'(z)\) so \(f'(x) - f(x) / x \geq 0\) for all \(x \geq 0\).
For \(x > 0\), \(g\) is differentiable and \(g'(x) = (xf'(x) - f(x)) / x^2 = (f'(x) - f(x) / x) / x \geq 0\), so \(g\) is monotonically increasing in \((0, \infty)\).

\section{Section 5 \#9}
Fix \(\epsilon > 0\).
There exists \(\delta > 0\) such that \(|x| < \delta\) implies \(|f'(x) - 3| < \epsilon\).
For all \(h > 0\), by mean value theorem there exists \(x \in (0, h)\) such that \(f'(x) = (f(h) - f(0)) / h\).
Likewise, for all \(h < 0\) again by mean value theorem there exists \(x \in (h, 0)\) such that \(f'(x) = (f(h) - f(0)) / h\).
Thus, if \(|h| < \delta\), there exists some \(0 < |x| < |h|\) such that \(f'(x) = (f(h) - f(0)) / h\), so we can write
\begin{align*}
  \left| \frac{f(h) - f(0)}{h} - 3 \right| = \left| f'(x) - 3 \right| < \epsilon
\end{align*}
and know that \(\lim_{h \to 0} (f(h) - f(0)) / h = 3\), so \(f'(0) = 3\).

\section{Section 5 \#11}
Since \(f''(x)\) exists, \(f'\) exists in a neighborhood of \(x\).
Applying theorem~5.13,
\begin{align*}
  \lim_{h \to 0} \frac{f(x + h) + f(x - h) - 2f(x)}{h^2}
  &= \lim_{h \to 0} \frac{f'(x + h) - f'(x - h)}{2h} \\
  &= \lim_{h \to 0} \frac{f'(x + h) - f'(x)}{2h} + \lim_{h \to 0} \frac{f'(x) - f'(x - h)}{2h} \\
  &= \frac{f''(x)}{2} + \frac{f''(x)}{2} = f''(x)
\end{align*}

Let \(f : \mathbb{R} \to \mathbb{R}\) as follows:
\begin{align*}
  f(x) = \begin{cases}
    x^2 & (x \geq 0) \\
    -x^2 & (x < 0)
  \end{cases}
\end{align*}
Then we can write
\begin{align*}
  \lim_{h \to 0} \frac{f(h) + f(-h) - 2f(0)}{h^2}
  = \lim_{h \to 0} \frac{h^2 + (-h^2)}{h^2} = 0
\end{align*}
However, since \(f'(x) = 2|x|\), \(f\) is not twice differentiable at 0.

\section{Section 5 \#14}
Suppose that \(f\) is convex, then for all \(s, t, u \in (a, b)\) such that \(s < t < u\),
\begin{align*}
  \frac{f(t) - f(s)}{t - s} \leq \frac{f(u) - f(s)}{u - s} \leq \frac{f(u) - f(t)}{u - t}
\end{align*}
Then we can write
\begin{align*}
  f'(s) = \lim_{t \to s} \frac{f(t) - f(s)}{t - s} \leq \lim_{t \to u} \frac{f(u) - f(t)}{u - t} = f'(u)
\end{align*}
so \(a < s < u < b\) implies \(f'(s) \leq f'(u)\) by sandwich theorem, and \(f'\) is monotonically increasing.

Now, suppose that \(f'\) is monotonically increasing.
For \(x, y \in (a, b)\) and \(\lambda \in (0, 1)\), if \(x < y\) then by mean value theorem there exists \(x_1 \in (x, \lambda x + (1 - \lambda) y)\) and \(x_2 \in (\lambda x + (1 - \lambda) y, y)\) such that
\begin{align*}
  \frac{f(\lambda x + (1 - \lambda) y) - f(x)}{[\lambda x + (1 - \lambda) y] - x} = f'(x_1)
  \leq f'(x_2) = \frac{f(y) - f(\lambda x + (1 - \lambda) y)}{y - [\lambda x + (1 - \lambda) y]}
\end{align*}
Then we can write
\begin{align*}
  \frac{f(\lambda x + (1 - \lambda) y) - f(x)}{(1 - \lambda)(y - x)}
  &\leq \frac{f(y) - f(\lambda x + (1 - \lambda) y)}{\lambda (y - x)} \\
  \lambda (f(\lambda x + (1 - \lambda) y) - f(x)) &\leq (1 - \lambda)(f(y) - f(\lambda x + (1 - \lambda) y)) \\
  f(\lambda x + (1 - \lambda) y) &\leq \lambda f(x) + (1 - \lambda) f(y)
\end{align*}
If \(x > y\), there exists \(x_1 \in (y, \lambda x + (1 - \lambda) y)\) and \(x_2 \in (\lambda x + (1 - \lambda) y, x)\) such that
\begin{align*}
  \frac{f(\lambda x + (1 - \lambda) y) - f(y)}{[\lambda x + (1 - \lambda) y] - y} = f'(x_1)
  \leq f'(x_2) = \frac{f(x) - f(\lambda x + (1 - \lambda) y)}{x - [\lambda x + (1 - \lambda) y]}
\end{align*}
Then we can also write
\begin{align*}
  \frac{f(\lambda x + (1 - \lambda) y) - f(y)}{\lambda (x - y)}
  &\leq \frac{f(x) - f(\lambda x + (1 - \lambda) y)}{(1 - \lambda)(x - y)} \\
  (1 - \lambda)(f(\lambda x + (1 - \lambda) y) - f(y)) &\leq \lambda (f(x) - f(\lambda x + (1 - \lambda) y)) \\
  f(\lambda x + (1 - \lambda) y) &\leq \lambda f(x) + (1 - \lambda) f(y)
\end{align*}
If \(x = y\), \(f(\lambda x + (1 - \lambda) y) \leq \lambda f(x) + (1 - \lambda) f(y)\) obviously holds, so \(f\) is convex.

Suppose that \(f''(x) \geq 0\) for all \(x \in (a, b)\).
Then, \(f'\) is a monotonically increasing function so \(f\) is convex.
If \(f\) is convex, \(f'\) is a monotonically increasing differentiable function.
Then, for all \(x \in (a, b)\) and \(h > 0\) such that \(x + h \in (a, b)\), we can write
\begin{align*}
  \frac{f'(x + h) - f'(x)}{h} \geq 0
\end{align*}
as \(f'\) is monotonically increasing.
By sandwich theorem, \(f''(x) = \lim_{h \to 0} (f'(x + h) - f(x)) / h \geq 0\) and we get the desired result.

\end{document}
